% !TeX spellcheck = en_US

\chapter{Introduction}
\label{chap: Introduction}
\section{Motivation}
European space industry has entered an economic era in which the funding availed to future space missions are not expected to grow significantly. At the same time, future missions are expected to achieve more and more challenging scientific goals with upcoming seasons of capped budgets for earth observation, scientific studies and space exploration \cite{PhdThesis}. This leads to a situation where the activities like mission analysis and system engineering will play a bigger role in the overall economy of the project, with a proportional increase of the time and cost invested in them \cite{PhdThesis}. The implication of this is that the realization activities, and software development among them are pushed forward in the project schedule and compressed \cite{SAVOIR}. Also, the complexity of the software product is foreseen to increase significantly to keep pace with rising mission needs, while the cost of the software development is expected to fit in the same or perhaps even decreased budget envelope. This situation is therefore calling for a rise in the cost effectiveness of the software development, thus ultimately increasing the "value" of the software product delivered with a given budget.

On-board software for satellites can be classified as high-integrity real-time software and the realization of the functional contents which add "value" to the product, is subject to stringent requirements at both process and product level in dimensions such as: time and space predictability, safety, dependability, security \cite{PhdThesis}. Also the software product is subject to extensive verification and validation steps to ascertain its quality \cite{SAVOIR}. In order to achieve all of this at reduced effort and a constant overall budget and at an acceptable level of quality, the concept of reusable software architecture plays a crucial role. In this context, a software architecture expresses an architectural framework that hosts the functional contents, architectural assumptions that and methodological principles that majorly contribute to the attainment of the desired quality of the software product \cite{PhdThesis}.

Parallely in the automotive domain, innovative vehicle functions are leading to a continual increase in the complexity of the vehicle architecture. At the same time, requirements are also sometimes contradictory, for example, supporting driver assistance systems in critical driving manoeuvers while also improving fuel economy and also conforming to the environmental standards \cite{AUTOSARurl}. Additional challenges include deeper integration of the infotainment and communication with the immediate vehicle environment and with online services. In order to continue to meet these requirements in the future, a new technological approach is required for the ECU software architecture \cite{AUTOSARurl}.  

More insights into the concepts of software architecture and a software reference architecture are given in the subsequent chapters.     

The initiatives in coming up with a software architecture in both space and automotive industries adopt the approaches based on the \ac{cbse} and \ac{mde}, which are in recent times gaining huge industrial acceptance in the domain of embedded real-time systems. This is not surprise at all since these two development paradigms promise important advantages such as better and more disciplined software design and increased reuse potential for the former; greater abstraction level and powerful automation capabilities for the latter \cite{CBSE}\cite{PhdThesis}. Many domain specific initiatives have shown that the the higher level of abstraction in the design process facilitated by the MDE allows addressing the non-functional concerns earlier in the development, thereby enabling proactive analysis, maturation and consolidation of the software design \cite{CompBasedDev}. Moreover, the automation capabilities of the MDE infrastructure may ease the generation of lower-level design artifacts and ease the generation of source code products.
Tasking first steps towards such an automation is the main motivation for this Master thesis.

\section{Thesis Structure}
This Master thesis is organized into following chapters:
\begin{description}
\item[Chapter~\ref{chap:OSRA} -- \nameref{chap:OSRA}:] This chapter gives an overall idea of a software architecture and a software reference architecture. It also explains the need for \ac{osra}, user needs and high level requirements which led to the development of OSRA and the overall software architectural concept defined in the development of OSRA.  
\item[Chapter~\ref{chap: Software development process} -- \nameref{chap: Software development process}:] This chapter explains in detail; the design and implementation steps for the \ac{cbse} approach which are enforced as a result of adoption of OSRA and its principles.
\item[Chapter~\ref{chap: Tasking framework} -- \nameref{chap: Tasking framework}:] This chapter explains the Tasking framework which is a portable framework for data flow and event driven cooperative multitasking.    
\item[Chapter~\ref{chap: Progamming model} -- \nameref{chap: Progamming model}:] This chapter aims at developing a programming model for this Master thesis. 
\item[Chapter~\ref{chap: Code generation} -- \nameref{chap: Code generation}:] This chapter deals with the mapping of \ac{obsw} model entities to the infrastructural code entities which are generated by a prototypical code generator, which is developed as a part of this Master thesis.
\item[Chapter~\ref{chap: code generator evaluation} -- \nameref{chap: code generator evaluation}:] This chapter subjects the prototypical code generator developed in this Master thesis to a tool acquisition evaluation using certain well defined evaluation methods. 
\item[Chapter~\ref{chap:conclusion} -- \nameref{chap:conclusion}:] This chapter contains discussions about the conclusions derived from the Master thesis and charts the future work plan and enhancements to this Master thesis.
\item[Appendix~\ref{chap: File structure} -- \nameref{chap: File structure}:] This section gives an idea of how the generated code for OBSW models can be organized into different files and folders.
\item[Appendix~\ref{chap: Extra examples} -- \nameref{chap: Extra examples}:] This section lists all the additional example \ac{obsw} models for which the code generator developed as a part of this Master thesis is tested.
\end{description}
