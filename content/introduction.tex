% !TeX spellcheck = en_US

\chapter{Introduction}
\label{chap: Introduction}
\section{Motivation}
European space industry has entered an economic era in which the funding availed to future space missions are not expected to grow significantly. At the same time, future missions are expected to achieve more and more challenging scientific goals with upcoming seasons of capped budgets for earth observation, scientific studies and space exploration \cite{PhdThesis}. This leads to a situation where the activities like mission analysis and system engineering will play a bigger role in the overall economy of the project, with a proportional increase of the time and cost invested on them \cite{PhdThesis}. The implication of this is that the realization activities, and software development among them are pushed forward in the project schedule and compressed \cite{SAVOIR}. Also, the complexity of the software product is foreseen to increase significantly to keep pace with rising mission needs, while the cost of the software development is expected to fit in the same or perhaps even decreased budget envelope. This situation is therefore calling for a rise in the cost effectiveness of the software development, thus ultimately increasing the "value" of the software product delivered with a given budget.

On-board software for satellites can be classified as high-integrity real-time software and the realization of the functional contents which add "value" to the product, is subject to stringent requirements at both process and product level in dimensions such as: time and space predictability, safety, dependability, security \cite{PhdThesis}. Also the software product is subject to extensive verification and validation steps to ascertain its quality \cite{SAVOIR}. In order to achieve all of this at reduced effort and a constant overall budget and at an acceptable level of quality, the concept of reusable software architecture plays a crucial role. In this context, a software architecture expresses an architectural framework that hosts the functional contents, architectural assumptions that and methodological principles that majorly contribute to the attainment of the desired quality of the software product \cite{PhdThesis}.

Parallely in the automotive domain, innovative vehicle functions are leading to a continual increase in the complexity of the vehicle architecture. At the same time, requirements are also sometimes contradictory, for example, supporting driver assistance systems in critical driving manoeuvers while also improving fuel economy and also conforming to the environmental standards \cite{AUTOSARurl}. Additional challenges include deeper integration of the infotainment and communication with the immediate vehicle environment and with online services. In order to continue to meet these requirements in the future, a new technological approach is required for the ECU software architecture \cite{AUTOSARurl}.  

More insights into the concepts of software architecture and a software reference architecture are given in the subsequent chapters     

The initiatives in coming up with a software architecture in both space and automotive industries adopt the approaches based on the \ac{cbse} and \ac{mde} which are in recent times gaining huge industrial acceptance in the domain of embedded real-time systems. This is not surprise at all since these two development paradigms promise important advantages such as better and more disciplined software design and increased reuse potential for the former; greater abstraction level and powerful automation capabilities for the latter \cite{CBSE}\cite{PhdThesis}. Many domain specific initiatives have shown that the the higher level of abstraction in the design process facilitated by the MDE allows addressing the non-functional concerns earlier in the development, thereby enabling proactive analysis, maturation and consolidation of the software design \cite{CompBasedDev}. Moreover, the automation capabilities of the MDE infrastructure may ease the generation of lower-level design artifacts and ease the generation of source code products.
First steps towards such an automation is the main motivation of this Master thesis.

\section*{Thesis Structure}
Die Arbeit ist in folgender Weise gegliedert:
\begin{description}
\item[Kapitel~\ref{chap:ch2} -- \nameref{chap:ch2}:] Hier werden werden die Grundlagen dieser Arbeit beschrieben.
\item[Kapitel~\ref{chap:conclusion} -- \nameref{chap:conclusion}] fasst die Ergebnisse der Arbeit zusammen und stellt Anknüpfungspunkte vor.
\end{description}
