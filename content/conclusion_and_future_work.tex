% !TeX spellcheck = en_US

\chapter{Results and Conclusions}
\label{chap:conclusion}

\section{Discussion}
As a part of this Master thesis, the following 
\begin{itemize}
\item A choice is made to use the Tasking framework as a computational model so that the OSRA component model statically binds to the Tasking framework which formally defines the computational entities and the rules which govern their usage 
\item A reference programming model is decided upon that enforces the analysis assumptions and which permits to express exclusively the semantics imposed by the analysis theory and which conveys the implementations of the desired non-functional properties using the primitives from the Tasking framework
\item Different corner cases which might arise during the construction of an OBSW model using the OSRA component model are identified 
\item An overall software design approach for the generated infrastructure code of the OBSW models is presented and a mapping of the OBSW model design entities to the infrastructural code entities is presented.  
\item A code generator is implemented, using which the generation of the entire non-functional code i.e., the code for handling the concurrency and interaction requirements for communication between components and generation of component containers and component connectors can be automated. The code generator uses the already tried and tested Tasking framework as the platform and bases the generated code on it. The advantage of this is that it eases the model-to-code transformation step 
\item The implemented code generator is tested for multiple OBSW models as shown in \cref{chap: Extra examples} which capture the different corner cases identified
\item For the simple OBSW model example which was introduced in \cref{chap: Code generation}, a set of unit test cases are written using Gtest and Gmock frameworks and the test coverage reports are generated
\end{itemize}

The following results were obtained 
\begin{itemize}
\item  The implemented code generator successfully generates the infrastructural code entities for all the example OBSW models listed in \cref{chap: Code generation} and in \cref{chap: Extra examples}. The generated code in all cases is successfully compiled to an executable along with Tasking framework using GCC C++ compiler conforming to the C++11 standard for the Linux platform
\item The test coverage reports generated for the unit tests written for the simple OBSW example are analyzed. The results show that the testability factor of the generated code is high and the infrastructural code entities can be efficiently tested, meeting the needs for high testability of the generated software      
\end{itemize}

\section{Identified shortcomings of Tasking framework}
During the course of the Master thesis, the following shortcomings of the current version of Tasking framework, which is chosen as a computational model for this Master thesis are identified:
\begin{itemize}
\item Tasks from Tasking framework are used in various threads of control as explained in \cref{chap: Progamming model}. At the heart of the Tasking framework is a scheduler which schedules tasks based on priorities and these tasks are non-preemptible at the moment \cite{TaskFr}. This is one of the critical shortcomings in the current version of the Tasking framework as it makes the generated software code which is based on Tasking framework not suitable for hard real-time systems \cite{TempIsolation}. Time-monitoring architectures such as Server-based architecture or Priority-Band architectures listed in \cite{TempIsolation} need to be adopted in the Tasking framework in order to make the generated code truly real-time capable. These architectures help in providing isolation of applications i.e., tasks (at least) along three orthogonal dimensional axes: time, space and communication \cite{TempIsolation}
\item In the current version of the Tasking framework there is no possibility to measure the run-time of the tasks and monitor deadline violations which are mostly caused by WCET overruns of either the task at hand or a higher priority task. This limits the extent of property preservation in the model-to-code transformation step \cite{TempIsolation}. It is of very high importance that the system properties asserted during the analysis and the assumptions made for the analysis to hold are preserved across implementation and execution \cite{EvoRAVCodeAr}\cite{TempIsolation}
\item It is also not possible to measure the execution time of a group of tasks which are associated to the single time budget so that the group budget is accounted for their collective execution time. This incapability makes the adoption of Server-based architecture in the Tasking framework more difficult \cite{TempIsolation}
\item In line with the inability to measure the run-time of the tasks from the Tasking framework, Tasking Framework also does not provide any constructs for at least coarse-grained fault detection and fault handling in case of deadline misses 
\end{itemize}

\section{Future Work}
As an enhancement to the current work, it is possible to extend this Master thesis in the following directions:
\begin{itemize}
\item A rather straight forward improvement would be to generate the unit test cases and mock classes for different infrastructural code entities automatically as part of the model-to-code transformation step. This helps in automating the testing of the generated code and testing the code entities independent of each other
\item The generated infrastructure code is found to have all the good characteristics of a software as listed in \cref{subsection: Software design approach} although the generated code needs to be rigorously evaluated for each of these characteristics 
\item Enable model-based round-trip analysis by conducting static analysis of the system model in the non-functional dimensions of interest. For example, schedulability analysis which verifies whether the timing non-functional requirements can be met \cite{ScheduAnaly},\cite{CompBasedProcess}.  For this, a platform specific model (PSM) i.e., a Schedulability Analysis Model (SAM) can be created from the declarative specification of the concurrent semantics that decorate the user model and inputs for the schedulability analysis tools can be automated and the results of the analysis can be seamlessly propogated back to the user space \cite{ScheduAnaly}
\item Model-to-code transformation of more complex data types such as opaque types, arrays, structures, unions etc which are possible to be instantiated in the OBSW model
\item Modeling of the relevant aspects of the hardware architecture and of the execution platform services are not considered in this Master thesis. This includes deciding upon a hardware topology which might include processing units and memory units, pseudo components for avionics equipments such as sensors, actuators, storage memories and remote terminals and hardware interconnections such as buses, point-to-point links, serial lines etc. The execution platform services such as Monitoring and control (M\&C) can be modeled as execution platform service instances and included in the hardware topology  
\item As only a small subset of possible non-functional properties are considered in this Master thesis, the realization of large number of possible non-functional properties are not considered. Non-functional properties such as worst-case execution time (WCET) bound for a certain operation, maximum memory footprint for a component implementation, communication budget allowed for an implementation, size of data types allowed in the communication etc., are yet to be effectively handled
\item Even though the programming model in \cref{chap: Progamming model} discusses about handling the non-functional property \texttt{Bursty} which can be set on the provided interface side of the component offering the service, this is not part of the current version of the code generator due to time constraints in the Master thesis       
\end{itemize} 
\label{section: Future work}

