% !TeX spellcheck = en_US

\chapter{Overall component-based software development process}
\label{chap: Software development process}
\section{Introduction}
In this chapter the design and implementation steps for the component-based software engineering (CBSE) approach are elaborated. The software design process involves two main actors: the software architect who is responsible for the entire software and provides support at system-level to the customer, and the software supplier who is responsible for the development of part of the software \cite{CompBasedProcess}. The parts of the software supplied by the software suppliers are then integrated in the final integration step.

Most of the activities described below come under the responsibility of the software architect, but as soon as the component is defined, it can undergo a detailed design and code implementation, it may indicate some shortcomings and flaws in the design of the component. That is when a re-design, re-negotiation of the component definition needs to be done, it often leads to a iterative/incremental development process \cite{ScheduAnaly}. Detailed design and implementation of components is usually done by the software developers are may be subcontracted to third party software suppliers. The references 

\section{Design entities and design steps}
There are two kind of entities which are defined in the architecture: Design-level entities which are explicitly specified in the design space and require the skills of the user to use them, real-time architecture entities which are not explicitly represented in the design space, instead they are automatically generated by the code-generation engines. The automatic generation of containers and connectors are possible when a particular computation model and execution platform are adopted (The choices made for this Master thesis are explained in the following chapters)  

The following entities belong to the design space: Data types, events, interfaces, component types, component implementations, component instances, component bindings and the entities for the description of the hardware topology and platforms. The following entities belong to the real-time architecture: containers and connectors.

The development process is clearly divided into different steps:

\begin{description}
\item [Step 1: Definition of data types and events] Data types are the basic entities in the approach and they can be primitive types, enumerations, ranged or constrained types, arrays or composite types (like structs in C or record types in Ada). An event is used in the publish-subscribe communication paradigm and it is an asynchronous message passing scheme.

\item [Step 2: Definition of interfaces] A set of operations with one or more already typed parameters, each with a direction (\texttt{in, in out, out}) are grouped together to form an interface. The interface can also hold a set of interface attributes of an already defined data type. The interface attributes can have read-only or read-write accesses. From the list of interface attributes, set of \texttt{getter} and \texttt{setter} operations can be generated for the attribute access, in particular \texttt{getter} operations for attributes with \texttt{read} access only and \texttt{getter},\texttt{setter} operations for attributes with \texttt{read-write} access. 

\item [Step 3: Definition of component types] Component types form the basis of reusable software asset. The software architect defines the component type to provide the specification of the functions that the component of this type would implement. The component types are independent of each other and they consist of:
\begin{itemize}
\item One or more provided interfaces, which list the services that the component of this type would provide
\item Required interfaces, which list the functional services that the component of this type would require in order to function correctly according to the functional specifications
\item A set of component type attributes of already defined data types. They are local to the component and they cannot be accessed from outside.
\item Event emitter/receiver ports to raise or receive events
In order to specify the provided and required interfaces, the component type references the interfaces that were defined in Step 2. This helps in straight forward matching of the required and provided interfaces  
\end{itemize}

\item [Step 4: Definition of component implementations] The software architect now creates and refines a component implementation from the component type. The component implementation contains the functional code in the form of source code that implements all the services that the component is supposed to provide. It acts as a  black box and only its external interfaces are only that matter. It is also a subcontracting unit to the software supplier. 

A component type can have more than one implementations and all of these implementations contain only pure sequential code and is void of any tasking or timing constructs. Implementations can be developed in multiple languages such as Ada, C, C++ etc.  

Component implementation should also provide constructs to store the attributes exposed through its provided interfaces and its component type. Technical budgets such as worst-case execution time (WCET) for a particular operation, maximum memory foot-print for component implementation, maximum number of calls to a certain operation on a required interface. This helps in determining the communication budget allocated because the size of data types used for communication is already known. Component implementation is thence a particularly attractive unit to be subcontracted to third-party because the software architect can define components, attached technical budgets to it and delegate the implementation to he third parties. The third party might add additional operations to the component implementation as and when necessary for the implementation.  

\item [Step 5: Definition of component instances] A component instance is an instance of a component implementation. It is the deployment unit subject to allocation on a processing unit and it is on which the non-functional properties are specified. Specifically, they are attached to the provided interface side of the component, as they are specifically the expression of a property or a provision of the component instance. There will not be any non-functional concerns attached to the required interface side of the component. 

\item [Step 6: Definition of component bindings] Component bindings, as the name suggests are the connections between one required interface of a component and the provided interface of another component and these bindings are set at design time and is subjected to static type matching to ensure that correct required and provided interfaces are connected to one another. This can be done by asserting the compatibility of the two interfaces (by type system or by inspection of the signature of their operations). If the binding is legal then whenever a call is made to an operation in the required interface, the call is dispatch to the correct operation in the bound provided interface. The signature of the call calling operation in the RI (required interface) and the called operation in the PI (provided interface) are different and the connector is in charge of performing this step and a tool support (possibly a back-end code generator) should help the configuration of the connector to perform this kind of binding.

It is also possible in this step to trace bindings between an event emitter port of one component and an event receiver port of another component.

\item [Step 7: Specification of non-functional attributes] After component instances and component bindings have been defined, the software architect adds non-functional attributes to the services of the provided interfaces. 

In this step, the software architect specifies the timing and the synchronization attributes. At first, the concurrency kind of the operation is established, and they can be \texttt{immediate} or \texttt{deferred} operations. In case of \texttt{immediate} operation, it is executed in the flow of control of the caller (synchronous) and in case of \texttt{deferred} operation, the operation is executed by a dedicated flow of control in the callee. 

An \texttt{immediate} operation is said to be \texttt{protected} if it needs to be protected from data races in case of concurrent calls and it is said to be \texttt{unprotected} if it is free from such risks. In case of a \texttt{deferred} operation type, the architect must chose one of the following release patterns of the operation:

\begin{description}
\item [Periodic operation] The execution platform executes the operation at fixed periods with a dedicated flow of control.
\item [Sporadic operation] Two subsequent execution requests are separated by a minimum timespan called the minimum inter-arrival time (MIAT). The execution platform and the infrastructural code  guarantees this MIAT separation between two subsequent calls to the operation and the component implementer does not have to worry about it.
\item [Bursty operation] Only particular number of activations of an operation is allowed in a bounded interval of time. Again the execution platform and the infrastructure code guarantees this and the component implementer does not have to worry about it as for the sporadic operation.
\end{description}

For all the operations which have concurrency as \texttt{deferred}, the softare architect must provide the worst case execution time (WCET) of the operation. A preliminary value of WCET is initially provided based on previous use of operations in other projects (if any) and they can be refined with bounds at later stages after performing a timing analysis for a given target platform.

The Component model of the OSRA also provides the software architect an option to define measurement units (conversion factors between them) and reuse the definitions across other projects. Non-functional attributes eg. Period of a deferred operation has a value and a measurement unit with it.

\item [Step 8: Definition of physical architecture] The hardware topology provides a description of the system hardware limited to the aspects related to communication, analysis and code generation. It also provides a model-level description of the relevant hardware of the system. In the hardware topology, following elements are described:
\begin{itemize}
\item Processing units that have a general-purpose processing capability
\item Avionics Equipment/Instruments/Remote terminals
\item The interconnection between the elements mentioned above 
\item A representation of the ground segment/other satellites (eg. Formation flying) to state the connection between the satellite and ground segment or other space segments
\end{itemize}		
For the specification of these elements, following attributes are used:
\begin{description}
\item [Processor frequency] This is used for processors to re-scale WCET values expressed in processor cycles in Step 6
\item [Bandwidth] This is used for buses and point-to-point links and it indicates maximum blocking time due to non-preemptability of the lower priority message transmission (for whatever reason), minimum and maximum size of packets, minimum and maximum propagation delay, the maximum time that the bus arbiter/driver needs to prepare and send a message on the physical channel and maximum time for the message to reach the receiver    
\end{description}

\item [Step 9: Component instances and component bindings deployment] In this step the component instances are allocated on the processing units defined in the hardware topology (refer Step 8). In majority of the cases, it is straight-forward to allocate the bindings between the components as they are deployed on the same processing units. In other cases, they need to be specifically allocated.

\item [Step 10: Model-based analysis] The system model developed within the software reference architecture is subjected to schedulability analysis to determine whether the timing requirements set o the interfaces can be met.

From the user model which is a PIM (Platform Independent Model), a Schedulability Analysis Model (SAM) which is a PSM (Platform Specific Model) is created. This model is subjected to analysis and the results of the analysis is available for the software architect as read-only result. 

The analysis transformation chain requires a model representation of the generated containers and connectors to be defined in the SAM for an accurate analysis.

Note: As the model based analysis requires a model representation of the containers and connectors, and as the topic of this master thesis is to generate containers and connectors, it is not possible to do an accurate model based schedulability analysis at the current phase of development. More about model based schedulability analysis done in another project ASSERT is available in the Appendix section.

\item [Step 11: Generation of containers and connectors]  This step is one of the main focus points of the master thesis. Containers and connectors are generated and they specify:
\begin{itemize}
\item The structure of each container in terms of the required and provided interfaces of the enclosed component that they delegate and subsume
\item The structure of each connector 
\end{itemize} 
The non-functional attributes and the component instance and component connector deployment play a major role in determining the creation of connectors and containers and how component instances and their operations are allocated to them.

Concurrency can be achieved by encapsulating sequential procedures into tasks which reside in containers and the protection from concurrent accesses can be provided by attaching them concurrency control structures. All of this can be achieved without modifying he sequential code and simply by following the use relations among the components.

In order for the OBSW to interact with the external world, sensors and actuators need to be provided. These hardware entities are represented as pseudo (which indicates that this component is for interaction purpose only) components and software capability is attached to these components at the component instance level.    
\end{description}

\section{Design flow and design views}
\label{section: Design flow and views}
When the component model is defined, it also defines implicitly a design flow that needs to be followed to be able to create an OBSW meet all its user needs and high level requirements and the design flow is as explained in the previous section. 

The component model is accompanied by the following design views:
\begin{description}
\item [Data view] This view is for the description of data types and events
\item [Component view] For definition of interfaces, components and the binding between them to fulfill their required needs
\item [Hardware view] For the specification of the hardware and the network topology
\item [Deployment view] For the allocation of components to computational nodes
\item [Non-functional view] In this view, the non-functional attributes are attached to the functional description of components
\item [Space-specific view] In this view, the services related to the commandability and observability of the spacecraft are specified
\end{description}  

\section{Language units of the OSRA}
The modeling language provided to the software architect to model the OBSW is divided for the ease of construction of the OBSW models into a set of language units. Each language unit consists of closely related metamodel entities. OSRA Component model is composed of the following language units:

\begin{description}
\item [\texttt{CommonKernel}] Defines the basic entities that are used as the base elements of the language architecture
\item [\texttt{DataTypes}] Defines all the possible data and the data types that can be used in an OBSW model
\item [\texttt{SCM Kernel}] Defines the infrastructural part of the Space Component Model (SCM) which can be considered as the language to express all the concerns expressed by an OBSW model
\item [\texttt{Component}] Defines a complete set of interfacing features (interfaces, events, datasets), component types and implementations, interface ports
\item [\texttt{Non-functional properties}] Defines the non-functional properties that can be applied to the modeling entities and a new language called the Value Specification Language (VSL) to specify values characterized by the measurement units
\item [\texttt{Deployment}] Defines instantiation and deployment entities such as component instances, connection between them and their deployment on the hardware architecture
\item [\texttt{Monitor and Control (M\&C)}] Defines the means to specify the technical properties related to M\&C that shall be provided in the OBSW model
\item [\texttt{Hardware execution platform}] Defines entities related to the execution platform, Time Space Partitioning (TSP) and the hardware architecture   
\end{description}

\section{OSRA SCM Model Editor}
\label{OSRA editor} 
The toolset that the software architect can use to build OBSW models is organised as a set of Eclipse features and Eclipse plugins. 

The toolset is available as
\begin{itemize}
\item A pre-installed Eclipse (Eclipse Neon) for Windows 64-bit  
\item An update site which consists of a set of static files which can be placed locally, on a web-server or on a file-server. 
\end{itemize}
In the latter case, the software architect would have to use the Eclipse Update Manager to install the plugins. 

In line with the design flow and design views explained in section \cref{section: Design flow and views}, different OSRA diagrams can be created with the help of OSRA SCM Model editor namely:
\begin{description}
\item [Interfaces, Events and Datasets diagram] This is the first diagram of the OSRA activity and allows to define the data types, events, datasets and the interfaces that would be used by the components in the Component Types diagram.
\item [Component Types diagram] This is the second diagram of the OSRA activity and it allows to define the component types, device types, execution platform service types, partition proxy types, required ports whose implementation would be used by the Component instances diagram
\item [Component Instance diagram] This is the third diagram of the OSRA activity and it allows to define the component instances, device instances, execution platform service instances, partition proxy instances, provided interface slots, data receiver slots and event receiver slots.
\item [Hardware diagram] This is the last diagram of the OSRA activity and it allows to define the hardware elements such as processor boards, mass memory, devices, buses.
\end{description}

There are also tables which are provided for diagram elements (if applicable) and they are usually found as tabs in pop-up window associated with the group that the element belongs to. Tables are usually classical tables where rows represents an element and each column represent a potentially computed property of the element. Rows can also contain sub-rows recursively which represent the sub-elements and the software architect can collapse or exapand these sub-elements as desired.    
    

  