% !TeX spellcheck = en_US
% Dieses Dokument muss mit PDFLatex gesetzt werden
% Vorteil: Grafiken koennen als jpg, png, ... verwendet werden
%          und die Links im Dokument sind auch gleich richtig
%
%Ermöglicht \\ bei der Titelseite (z.B. bei supervisor)
%Siehe https://github.com/latextemplates/uni-stuttgart-cs-cover/issues/4
\RequirePackage{kvoptions-patch}

%Englisch:
\let\ifdeutsch\iffalse
\let\ifenglisch\iftrue{}

%Deutsch:
%\let\ifdeutsch\iftrue
%\let\ifenglisch\iffalse

%
\ifdeutsch
	\PassOptionsToClass{numbers=noenddot}{scrbook}
\fi

%Warns about outdated packages and missing caption delcarations
%See https://www.ctan.org/pkg/nag
\RequirePackage[l2tabu, orthodox]{nag}

%Neue deutsche Trennmuster
%Siehe http://www.ctan.org/pkg/dehyph-exptl und http://projekte.dante.de/Trennmuster/WebHome
%Nur für pdflatex, nicht für lualatex
\RequirePackage{ifluatex}
\ifluatex
%do not load anything
\else
	\ifdeutsch
		\RequirePackage[ngerman=ngerman-x-latest]{hyphsubst}
	\fi
\fi

\documentclass[
               fontsize=12pt, %Default: 11pt, bei Linux Libertine zu klein zum Lesen
% BEGINN: Optionen für typearea
               paper=a4,
               twoside, % fuer die Betrachtung am Schirm ungeschickt
               BCOR=3mm, % Hack für BCOR (1.92 o.ä.), da bei BCOR2mm die Fuellpunkte beim Inhaltsverzeichnis falsch sind. Hack aber nicht mehr nötig: microtype für Verzeichnisse ausschalten hilft.
               DIV=13,   % je höher der DIV-Wert, desto mehr geht auf eine Seite. Gute werde sind zwischen DIV=12 und DIV=15
               headinclude=true,
               footinclude=false,
% ENDE: Optionen für typearea
%               titlepage,
               bibliography=totoc,
%               idxtotoc,   %Index ins Inhaltsverzeichnis
%                liststotoc, %List of X ins Inhaltsverzeichnis, mit liststotocnumbered werden die Abbildungsverzeichnisse nummeriert
               headsepline,
               cleardoublepage=empty,
               parskip=half,
%               draft    % um zu sehen, wo noch nachgebessert werden muss - wichtig, da Bindungskorrektur mit drin
               final   % ACHTUNG! - in pagestyle.tex noch Seitenstil anpassen
               ]{scrbook}

\input{preambel/packages_and_options}

%Der untere Rand darf "flattern"
\raggedbottom

%%%
% Wie tief wird das Inhaltsverzeichnis aufgeschlüsselt
% 0 --\chapter
% 1 --\section % fuer kuerzeres Inhaltsverzeichnis verwenden - oder minitoc benutzen
% 2 --\subsection
% 3 --\subsubsection
% 4 --\paragraph
\setcounter{tocdepth}{1}
%
%%%

\makeindex

%Angaben in die PDF-Infos uebernehmen
\makeatletter
\hypersetup{
            pdftitle={\doctitle}, %Titel der Arbeit
            pdfauthor={\docauthor}, %Author
            pdfkeywords={}, % CR-Klassifikation und ggf. weitere Stichworte
            pdfsubject={}
}
\makeatother


%%% acro
% alle Acronyme die verwendet werden kommen hier her.

\DeclareAcronym{ER}{short = ER , long = error rate}
\DeclareAcronym{FR}{short = FR , long = Fehlerrate}

%%%


\begin{document}

%tex4ht-Konvertierung verschönern
\iftex4ht
% tell tex4ht to create picures also for formulas starting with '$'
% WARNING: a tex4ht run now takes forever!
\Configure{$}{\PicMath}{\EndPicMath}{} 
%$ % <- syntax highlighting fix for emacs
\Css{body {text-align:justify;}}

%conversion of .pdf to .png
\Configure{graphics*}  
         {pdf}  
         {\Needs{"convert \csname Gin@base\endcsname.pdf  
                               \csname Gin@base\endcsname.png"}%  
          \Picture[pict]{\csname Gin@base\endcsname.png}%  
         }  
\fi

%Tipp von http://goemonx.blogspot.de/2012/01/pdflatex-ligaturen-und-copynpaste.html
%siehe auch http://tex.stackexchange.com/questions/4397/make-ligatures-in-linux-libertine-copyable-and-searchable
%
%ONLY WORKS ON MiKTeX
%On other systems, download glyphtounicode.tex from http://pdftex.sarovar.org/misc/
%
%input glyphtounicode.tex
\pdfgentounicode=1

\VerbatimFootnotes %verbatim text in Fußnoten erlauben. Geht normalerweise nicht.

\input{macros/commands}
\pagenumbering{roman}
\Titelblatt

%Eigener Seitenstil fuer die Kurzfassung und das Inhaltsverzeichnis
\deftripstyle{preamble}{}{}{}{}{}{\pagemark}
%Doku zu deftripstyle: scrguide.pdf
\pagestyle{preamble}
\renewcommand*{\chapterpagestyle}{preamble}

\section*{Abstract}
European Space Agency (ESA) and its industrial partners have come up with the On-board Software Reference Architecture (OSRA) with the aim of favoring the adoption of a software reference architecture across their software supply chain. The center of that strategy involves around a component model called the Space Component Model (SCM) and the software development process that builds on it. The SCM aims to model application software as a set of independent software components which interact with each other via clearly defined interfaces with certain guarantees. The SCM is present as an EMF based Ecore meta model and it comes with a graphical editor called the OSRA SCM Model editor. Although the SCM provides information about how components interact with each other through the provided or required services, it does not provide an implementation of those services. The work presented here in this Master thesis aims at implementing a back-end code generator for OSRA, supporting the general vision that in the future, an application developer would create and configure components for his/her on-board application and capture the desired component interactions in an SCM model instance. He/she can then generate code skeletons for the model, i.e., all the concurrency behavior, data exchange, type conversion, etc. automatically using the code generator. As a result, the developer can only concentrate on implementing the functional code of each on-board software component, which in-turn results in shorter development cycles and high cost-efficiency. The generated code uses Tasking Framework as a well-formed platform and bases the generated code on it. The Tasking Framework is a portable framework for data flow and event drive cooperative multitasking which is written in a safe subset of C++. It is developed by the group 'On-board Software Systems' of the German Aerospace Center (DLR) department of Software for Space Systems and Interactive Visualization.      
\cleardoublepage

% BEGIN: Verzeichnisse

\iftex4ht
\else
\microtypesetup{protrusion=false}
\fi

%%%
% Literaturverzeichnis ins TOC mit aufnehmen, aber nur wenn nichts anderes mehr hilft!
% \addcontentsline{toc}{chapter}{Literaturverzeichnis}
%
% oder zB
%\addcontentsline{toc}{section}{Abkürzungsverzeichnis}
%
%%%

%Produce table of contents
%
%In case you have trouble with headings reaching into the page numbers, enable the following three lines.
%Hint by http://golatex.de/inhaltsverzeichnis-schreibt-ueber-rand-t3106.html
%
%\makeatletter
%\renewcommand{\@pnumwidth}{2em}
%\makeatother
%
\tableofcontents

% Bei einem ungünstigen Seitenumbruch im Inhaltsverzeichnis, kann dieser mit
% \addtocontents{toc}{\protect\newpage}
% an der passenden Stelle im Fließtext erzwungen werden.

\listoffigures
\listoftables

\ifdeutsch
\printacronyms[name=Abkürzungsverzeichnis, heading=chapter*]
\else
\printacronyms[name=List of Acronyms, heading=chapter*]
\fi

%Wird nur bei Verwendung von der lstlisting-Umgebung mit dem "caption"-Parameter benoetigt
%\lstlistoflistings 
%ansonsten:
\ifdeutsch
\listof{Listing}{Verzeichnis der Listings}
\else
\listof{Listing}{List of Listings}
\fi

%mittels \newfloat wurde die Algorithmus-Gleitumgebung definiert.
%Mit folgendem Befehl werden alle floats dieses Typs ausgegeben
\ifdeutsch
\listof{Algorithmus}{Verzeichnis der Algorithmen}
\else
\listof{Algorithmus}{List of Algorithms}
\fi
%\listofalgorithms %Ist nur für Algorithmen, die mittels \begin{algorithm} umschlossen werden, nötig

\iftex4ht
\else
%Optischen Randausgleich und Grauwertkorrektur wieder aktivieren
\microtypesetup{protrusion=true}
\fi

% END: Verzeichnisse

\mainmatter
\pagenumbering{arabic}

\renewcommand*{\chapterpagestyle}{scrplain}
\pagestyle{scrheadings}
\input{preambel/pagestyle}
%
%
% ** Hier wird der Text eingebunden **
%
% !TeX spellcheck = en_US

\chapter{Introduction}
\label{chap: Introduction}
\section{Motivation}
European space industry has entered an economic era in which the funding availed to future space missions are not expected to grow significantly. At the same time, future missions are expected to achieve more and more challenging scientific goals with upcoming seasons of capped budgets for earth observation, scientific studies and space exploration \cite{PhdThesis}. This leads to a situation where the activities like mission analysis and system engineering will play a bigger role in the overall economy of the project, with a proportional increase of the time and cost invested in them \cite{PhdThesis}. The implication of this is that the realization activities, and software development among them are pushed forward in the project schedule and compressed \cite{SAVOIR}. Also, the complexity of the software product is foreseen to increase significantly to keep pace with rising mission needs, while the cost of the software development is expected to fit in the same or perhaps even decreased budget envelope. This situation is therefore calling for a rise in the cost effectiveness of the software development, thus ultimately increasing the "value" of the software product delivered with a given budget.

On-board software for satellites can be classified as high-integrity real-time software and the realization of the functional contents which add "value" to the product, is subject to stringent requirements at both process and product level in dimensions such as: time and space predictability, safety, dependability, security \cite{PhdThesis}. Also the software product is subject to extensive verification and validation steps to ascertain its quality \cite{SAVOIR}. In order to achieve all of this at reduced effort and a constant overall budget and at an acceptable level of quality, the concept of reusable software architecture plays a crucial role. In this context, a software architecture expresses an architectural framework that hosts the functional contents, architectural assumptions and methodological principles that majorly contribute to the attainment of the desired quality of the software product \cite{PhdThesis}.

Parallely in the automotive domain, innovative vehicle functions are leading to a continual increase in the complexity of the vehicle architecture. At the same time, requirements are also sometimes contradictory, for example, supporting driver assistance systems in critical driving manoeuvers while also improving fuel economy and also conforming to the environmental standards \cite{AUTOSARurl}. Additional challenges include deeper integration of the infotainment and communication with the immediate vehicle environment and with online services. In order to continue to meet these requirements in the future, a new technological approach is required for the ECU software architecture \cite{AUTOSARurl}.  

More insights into the concepts of software architecture and a software reference architecture are given in the subsequent chapters.     

The initiatives in coming up with a software architecture in both space and automotive industries adopt the approaches based on the \ac{cbse} and \ac{mde}, which are in recent times gaining huge industrial acceptance in the domain of embedded real-time systems. This is not surprise at all since these two development paradigms promise important advantages such as better and more disciplined software design and increased reuse potential for the former; greater abstraction level and powerful automation capabilities for the latter \cite{CBSE}\cite{PhdThesis}. Many domain specific initiatives have shown that the the higher level of abstraction in the design process facilitated by the MDE allows addressing the non-functional concerns earlier in the development, thereby enabling proactive analysis, maturation and consolidation of the software design \cite{CompBasedDev}. Moreover, the automation capabilities of the MDE infrastructure may ease the generation of lower-level design artifacts and ease the generation of source code products.
Taking first steps towards such an automation is the main motivation for this Master thesis.

\section{Thesis Structure}
This Master thesis is organized into following chapters:
\begin{description}
\item[Chapter~\ref{chap:OSRA} -- \nameref{chap:OSRA}:] This chapter gives an overall idea of a software architecture and a software reference architecture. It also explains the need for \ac{osra}, user needs and high level requirements which led to the development of OSRA and the overall software architectural concept defined in the development of OSRA.  
\item[Chapter~\ref{chap: Software development process} -- \nameref{chap: Software development process}:] This chapter explains in detail; the design and implementation steps for the \ac{cbse} approach which are enforced as a result of adoption of OSRA and its principles.
\item[Chapter~\ref{chap: Tasking framework} -- \nameref{chap: Tasking framework}:] This chapter explains the Tasking framework which is a portable framework for data flow and event driven cooperative multitasking.    
\item[Chapter~\ref{chap: Progamming model} -- \nameref{chap: Progamming model}:] This chapter aims at developing a programming model for this Master thesis. 
\item[Chapter~\ref{chap: Code generation} -- \nameref{chap: Code generation}:] This chapter deals with the mapping of \ac{obsw} model entities to the infrastructural code entities which are generated by a prototypical code generator, which is developed as a part of this Master thesis.
\item[Chapter~\ref{chap: code generator evaluation} -- \nameref{chap: code generator evaluation}:] This chapter subjects the prototypical code generator developed in this Master thesis to a tool acquisition evaluation using certain well defined evaluation methods. 
\item[Chapter~\ref{chap:conclusion} -- \nameref{chap:conclusion}:] This chapter contains discussions about the conclusions derived from the Master thesis and charts the future work plan and enhancements to this Master thesis.
\item[Appendix~\ref{chap: File structure} -- \nameref{chap: File structure}:] This section gives an idea of how the generated code for OBSW models can be organized into different files and folders.
\item[Appendix~\ref{chap: Extra examples} -- \nameref{chap: Extra examples}:] This section lists all the additional example \ac{obsw} models for which the code generator developed as a part of this Master thesis is tested.
\end{description}

%\input{...further chapters...}
% !TeX spellcheck = en_US

\chapter{The On-board Software Reference Architecture (OSRA)}
\label{chap:OSRA}
\section*{Introduction}
\subsection*{Background}
Space industry has recognized already for quite some time the need to raise the level of standardisation in the avionics system in order to increase the efficiency and reduce cost and schedule in the development.
The implementation of such a vision is expected to provide benefits for all the stake-holders in the space community:
\begin{description}
\item  [Customer Agencies] Significant drop in the project development lifecycle and the risk involved in the software development
\item [System Integrators] There would be increased competition amongst them to deliver at lower price and maintain shorter time-to market and there would also be multi-supplier option
\item [Supplier Industry] Benefits from diversified customer bases and the supplied building blocks would be compatible with prime architectures across the board
\end{description}

Similar initiatives have already been taken across various industries and eg. AUTOSAR for the automotive industry is worthy mentioning. Space can benefit from these examples by studies related to how these similar initiatives were successfully conducted and how they fared. ALthough the business model is different in the automotive and the space sectors, AUTOSAR demonstrates the need fir standardization is the key irrespective of the sector and is driven by the need of the industry to become more competitive.

Space primes and on-board software companies have made significant progress and have implemented and/or are implementing reuse on the basis of company's internal software refernce architectures and building blocks. However in for this standardization to provide maximum benefits, it has to be tackled at the European level rather than at company level.

ESA through its two parallel activities aimed at increasing the software reuse in on-board softwares (CORDET and Domeng) have confirmed that interface standardization allows to efficiently compose the software on the basis of existing and mature building blocks.

To refer to all ongoing initiatives and to provide a platform for technical discussions, related to the vision of avionics development through maximizing reuse and standardization, a "Space Avionics Open Interface Architecture" Advisory Group (SAVOIR Advisory Group) was created. SAVIOR Advisory Group decided to spawn a specific subgroup on-board software reference architectures called "SAVOIR Fair Architecture and Interface Reference Elaboration" working group (SAVOIR FAIRE). OSRA is the result of R\&D activities of this group.   

The On-board software reference architecture (OSRA) is designed to be a single, common and agreed framework for the definition of the on-board software (OBSW) of the future European Space Agency (ESA) missions. It is based on solid scientific foundations and accompanied by development methodology and architectural practices that fit the domain. A single software system would thus be an "instantiation" of the reference architecture to specific mission needs.

The software architecture is the key to create "good quality" software because it promotes architectural best practices and contributes to the quality of the software. A bad architecture hinders the fulfillment of functional, behavioral, non-functional and life-cycle requirements. Elevating a software architecture to software reference architecture permits to gather and re-use lessons learned and architectural best practices, give new projects a consolidated running start and promote a product line approach.

\section*{Need for reference architecture}
\subsection*{Motivation} 
The schedules of space projects are always decreasing and the team need to increase their efficiency and cost effectiveness in the development process of on-board avionics. But the on-board software is getting more complex because of the trend towards more functionality being implemented by the on-board software. Therefore the overall objective of space industry is now to standardize the avionics systems and therefore the on-board software.

A building block approach is one of the ways to tackle this problem. In this approach, the on-board software is implemented from a set of pre-developed and fully compatible building blocks, plus specific adaptations and "missionisation" according to specific mission requirements. The target missions are the core ESA missions, ie.e high reliability and availability spacecraft driven systems (eg. operational missions, science missions).

The "right" building blocks need to be produced and supplied by the suppliers to any system integrator and to achieve this, reference architectures need to be defined

Separation of the application aspects from the general-purpose data processing aspects is the key to generic/reusable software architectures. The lower layers of the architectures usually handle the implementation of communication, real time capabilities etc and the higher level layers usually deal with the application aspects. However there have to be ways to annotate the application building blocks (ABB) with sufficient information regarding requirements related to communication, real-time, dependability etc., so that the platform building blocks (PBB) can provide the suitable complete implementation. Development of interface specifications with reference architectures as the basis allows the implementation of the famous AUTOSAR concept: "Cooperate on standards, compete on implementation"

\subsection*{User needs}
These are some of the needs that were assimilated to guide the development of the software reference architecture:
\begin{description}
\item [Shorter software development time] The software development schedule should be reduced because usually the definition of the software requirements is done at a later stage and the final version of the software is expected to be released earlier. Even though the cost of the software itself us a minor fraction of the cost of the whole system in space industry, the impact of delays in availability of the software may have a huge impact on the overall schedule and consequently on the cost of the project
\item [Reduce recurring costs] It is important to identify and reduce the recurring costs and in turn help to use the project resources to focus on value added to the product or to reduce the cost of development while providing the same set of functions. Examples for recurring costs include device drivers, real-time operating system, providing communication services etc and it is important to note that these cots drivers do not provide an added value and are not mission specific.
\item [Quality of the product] The level of the quality (timing predictability, dependability of the software etc) of the software must at least be the same as the one of OBSW developed with current approaches.
\item [Increase cost-efficiency] Cost-efficiency is the "value" of the software product that is developed with a certain amount of budget. An increased cost-efficiency is achieved by developing the same set of functions for less budget, developing the same set of functions with more stringent requirements for the same budget and increasing the number of realized functions for the same budget. The budget available for the software development is not expected to grow and it may be indeed be subjected to reduction and hence new development approaches may be required to fulfill this user need. On contrary the performance of the application building blocks eg. accuracy of the AOCS controls is expected to grow and new complex functionalities are expected to develop
\item [Reduce Verfication and Validation effort] The main contributor to the cost of software development are V\&V activities which contribute anywhere between 50\% to 70\% of the overall cost. Adoption of the principle of Correction by Construction (C-by-C) which is one of the founding principles of choice (refer \cref{chap:OSRA}), analysis at early design stage and provisions for re-usability of (functional) tests are expected to reduce the Verification and Validation efforts. This also leads to shorter software development times and reduced costs. 
\item [Mitigate the impact of late requirement definition or change] Late refinement of system design, evolution of the operational level, late finalization of the system FDIR, software modification to compensate the problems in the hardware found during system integration may often lead to definition of new requirements or their changes for the software, anytime in the entire SW life-cycle.
\item [Support for various system integration strategies] Preliminary software releases are important to allow early system integration and software development may be managed with different strategies. It is necessary to respect these strategies and help final integration of increments or elements.
\item [Simplification and harmonization of FDIR] Simplification and coordination of the Fault Detection, Isolation and Recovery (FDIR) needs to be handled at both the system and software level. System engineers and software implementers need to justify the definition of the FDIR strategy at the system level and the software level respectively. A set of functionalities and design patterns need to be provided at the software level that cater to necessary mechanisms for the software realization of FDIR strategy
\item [Optimize flight maintenance] Flight maintenance may be required to change the OBSW and provision of the required operations and coordination of the strategy to perform it will decrese the time and cost of maintenance. It is better if parts of the software could be updated without having to reboot the CDMU. 
\item [Industrial policy support] The development process should enable multi-team software development. It is necessary to incorporate certain flexibility in the allocation of the software elements to industry, according to certain criterion such as prime/non-prime, or geographical return. Multi-team software development is essential to subcontract to non-primes while be in charge of the integration and apply the geographical return policy.
\item [Role of software suppliers] As discussed before, the new approach must increase the competence of the supplier and foster competition amongst the suppliers: Different suppliers may develop the same component and compete on quality, extension features, performance, cost and schedule. The suppliers will also profit from this approach as they do not have to adapt the software to specific development policies if each single prime.
\item [Dissemination activities] System engineers can be exposed to core principle of the process and if they derive specifications for the system out of the domain of reuse, the costs will certainly increase.
\item [Future needs] The trend of increasing complexity of the OBSW gives rise to several needs and these needs need to be subjected to evaluation and their impact on the software reference architecture needs to be monitored. Some of the examples of the future needs include integration of functions of different criticality and security levels, use of Time and Space Partitioning (TSP), support to the multi-core processors, contextual verification of safety properties.     
\end{description} 

\subsection*{High level requirements}
The user needs were translated into a set of high-level requirements:
\begin{description}
\item [Software reuse] The architecture shall be designed in such a way that the reuse of the functional aspects should be independent of the reuse of the non-functional aspects, reuse of the the unit, integration and validation tests by providing  a pre-qualification data package supported by a SW Reuse File in the sense of ECSS software standards. Traced to user needs:
\begin{itemize}
\item Shorter software development time
\item Reduce recurring costs
\end{itemize} 
\item [Separation of concerns] Separation of concerns is one of the cornerstone principles and it deals wit separating different aspects of the software design, in particular the functional and non-functional concerns. Separation of concerns helps to reuse functional concerns independently from non-functional concerns and hence increasing the software reuse. Traced to user needs:
\begin{itemize}
\item Quality of the product
\item Reduce Verification and Validation effort
\item Role of software suppliers 
\end{itemize}  
\item [Reuse of V\&V tests] The chosen architectural approach should also promote the reuse of Verification and Validation tests that were performed on the software and not just the software itself. The aim is to maximize the reuse of the tests written for the functional part of the component software. Traced to user needs:
\begin{itemize}
\item Shorter software development time
\item Reduce Verification and Validation effort 
\end{itemize}
\item [HW/SW Independence] It enables development of the software independent from the hardware features. Its is necessary to separate parts of the software that interact directly with the hardware into separate modules and make them accessible through defined interfaces. In this way, as long as the interface does not change, the software isolated from the changes in the hardware-dependent layer. Traced to user-needs:
\begin{itemize}
\item Quality of the product
\item Mitigate the impact of later requirements definition or change 
\item Support for various system integration changes
\end{itemize}	  	
\item [Component based approach] The whole software is designed as a composition of components that are reusable in nature. The architecture shall respect preservation of properties of individual building blocks once integrated into the architecture and it should be possible to calculate the system's property as a function of components' individual properties. The former is called composability and the latter is called compositionality. Traced to user needs:
\begin{itemize}
\item Shorter software development time
\item Reduce recurring costs
\item Increase cost-efficiency
\item Support for various system-integration changes
\item Product policy
\item Role of software suppliers 
\end{itemize}  
\item [Sofwtare observability] The software architecture should provided means to observe the software specific parts and extract current and past status of the software using the services specified by its operational scenarios. This prevents the need for post launch updates or patches of the software in case of failure analysis needs. Traced to user needs:
\begin{itemize}
\item Quality of the product
\item Reduce Verification and Validation effort
\item Simplification and harmonization of FDIR
\item Optimize flight maintenance 
\end{itemize} 
\item [Software analysability] he design process and methodology used for the reference architecture shall support the verification at design time of functional and non-functional properties. Traced to user needs:
\begin{itemize}
\item Quality of the product
\item Reduce Verification and Validation effort 
\end{itemize}
\item [Property preservation] The non-functional properties become the constraints on the system as they specify the "frame" in which the system is expected to behave and be consistent with what was predicted during the analysis. 
These properties have to be preserved or enforced so that these properties are not only used for the analysis of the software model, but also find their way through to the final system at run-time. Adequate mechanisms should be provided to handle the enforcement of properties and reactions to violation of the properties. Traced to user needs:
\begin{itemize}
\item Quality of the product
\item Reduce Verification and Validation effort
\end{itemize} 
\item [Integration of software building blocks] The architecture should allow the combination of coherent building blocks
\begin{itemize}
\item Shorter software development time
\item Mitigate the impact of late requirement definition or change
\item Support for various system integration strategies
\item Product policy
\item Role of software suppliers 
\end{itemize}
\item [Support for variability factors] In order to reduce the complexity of the architecture, the potential variation of the architecture induced by the variation of the domain must be isolated in some places such as reuse is improved and need for modification is decreased. Traced to user needs:
\begin{itemize}
\item Increase cost-efficiency 
\end{itemize}
\item [Late incorporation of modification in the software] The architecture should be immune to modification of the software late in the software life cycle. System integration always finds some system problems and it is the responsibility of the software to contain these problems and implement new requirements. The architecture to which the software is conformal to, should be able to handle these late modifications in the software.Traced to user needs:
\begin{itemize}
\item Mitigate the impact of late requirement definition or change 
\end{itemize}
\item [Provision of mechanisms for FDIR] The aircraft dependability should be handled by the architecture and in particular the Fault Detection, Isolation and Recovery. Traced to user needs:
\begin{itemize}
\item Simplification and harmonization of FDIR  
\end{itemize}
\item [Sofwtare update at run-time] The reference architecture should allow update to single software components as well as their bindings without having to reboot the entire on-board computer as it is a risk for the system and reduces the mission availability/uptime. Traced to user needs:
\begin{itemize}
\item Optimize flight maintenance 
\end{itemize}
\end{description} 

OSRA comprises of three layers: the component layer, the interaction layer and the execution platform
\texttt{\textbackslash{}chapter} benutzen oder pro Kapitel eine eigene Datei anlegen und \texttt{ausarbeitung.tex} anpassen.

LaTeX-Hinweise stehen in \cref{chap:latextipps}.

%noch etwas Fülltext
\blinddocument

% !TeX spellcheck = en_US

\chapter{Overall component-based software development process}
\label{chap: Software development process}
\section{Introduction}
In this chapter the design and implementation steps for the component-based software engineering (CBSE) approach are elaborated. The software design process involves two main actors: the software architect who is responsible for the entire software and provides support at system-level to the customer, and the software supplier who is responsible for the development of part of the software \cite{CompBasedProcess}. The parts of the software supplied by the software suppliers are then integrated in the final integration step.

Most of the activities described below come under the responsibility of the software architect, but as soon as the component is defined, it can undergo a detailed design and code implementation. This may indicate some shortcomings and flaws in the design of the component, which might trigger a re-design, re-negotiation of the component definition. This often leads to an iterative/incremental development process \cite{ScheduAnaly}. Detailed design and implementation of components are usually done by the software developers or it may be subcontracted to third party software suppliers. 

\section{Design entities and design steps}
\label{section: Design steps}
There are two kind of entities which are defined in OSRA: Design-level entities which are explicitly specified in the design space and require the skills of the user to use them, real-time architecture entities which are not explicitly represented in the design space, instead they are automatically generated by the code-generation engines. The automatic generation of containers and connectors are possible only upon the knowledge of the computation model and execution platform that are going to be adopted \cite{SAVOIR,CompBasedProcess}. As already mentioned in the previous chapter, this master thesis considers Tasking Framework as the computational model and a normal linux based machine as the execution platform.   

The following entities belong to the design space: Data types, events, interfaces, component types, component implementations, component instances, component bindings and the entities required for the description of the hardware topology and platforms. The following entities belong to the real-time architecture: containers and connectors.

The development process is clearly divided into different steps \cite{CompBasedProcess, PhdThesis,SAVOIR}:

\begin{description}
\item [Step 1: Definition of data types and events] Data types are the basic entities in the approach and they can be primitive types, enumerations, ranged or constrained types, arrays or composite types (like structs in C or record types in Ada). An event is used in the publish-subscribe communication paradigm and it is an asynchronous message passing scheme. 

\item [Step 2: Definition of interfaces] A set of operations with one or more already typed parameters, each with a direction (\texttt{in, in out, out}) are grouped together to form an interface. The interface can also hold a set of interface attributes of an already defined data type. The interface attributes can have read-only or read-write accesses. From the list of interface attributes, set of \texttt{getter} and \texttt{setter} operations can be generated for the attribute access, in particular \texttt{getter} operations for attributes with \texttt{read} only access and \texttt{getter},\texttt{setter} operations for attributes with \texttt{read-write} access. 

\cref{fig: Datatypes events and interfaces} depicts three data types and an event. Interface \texttt{AOCS\_IF} and \texttt{THR\_IF} implement only operations while interface \texttt{GYR\_IF} comprises one read-only attribute.

\begin{figure}[h]
	\centering
	\includegraphics[width=0.8\textwidth]{DatatypesEventsInterfaces.pdf}
	\caption{Data types, events and interfaces}
	\source{\cite{CompBasedProcess}}
	\label{fig: Datatypes events and interfaces}
\end{figure}

\item [Step 3: Definition of component types] Component types form the basis of a reusable software asset \cite{CompBasedProcess}. The software architect defines the component type to provide the specification of the functions that the component of this type would implement. The component types are independent of each other and they can consist of:
\begin{itemize}
\item One or more provided interfaces, which list the services that the component of this type would provide
\item One or more required interfaces, which list the functional services that the component of this type would require in order to function correctly according to the functional specifications
\item A set of component type attributes of already defined data types that are local to the component cannot be accessed from outside.
\item Event emitter/receiver ports to raise or receive events
In order to specify the provided and required interfaces, the component type references the interfaces that were defined in Step 2. This helps in straight forward matching of the required and provided interfaces  
\end{itemize}

\cref{fig: Component type} depicts a component type \texttt{AOCS}. This component type provides interface \texttt{AOCS\_IF} and requires interfaces \texttt{GYR\_IF} and \texttt{THR\_IF}. It also raises events of type \texttt{GYR\_FAILURE}. 

\begin{figure}[h]
	\centering
	\includegraphics[width=0.8\textwidth]{ComponentType.pdf}
	\caption{Component type}
	\source{\cite{CompBasedProcess}}
	\label{fig: Component type}
\end{figure}

\item [Step 4: Definition of component implementations] The software architect now creates and refines a component implementation from the component type. The component implementation contains the functional code in the form of source code that implements all the services that the component is supposed to provide. It acts as a  black box and only its external interfaces are only that matter. It is also a subcontracting unit to the software supplier. 

A component type can have more than one implementation and all of these implementations contain only pure sequential code i.e. it is void of any tasking or timing constructs. Implementations can be developed in multiple languages such as Ada, C, C++ etc.  

The component implementation should also provide constructs to store the attributes exposed through its provided interfaces and its component type. Technical budgets such as worst-case execution time (WCET) for a particular operation, maximum memory foot-print for component implementation, maximum number of calls to a certain operation on a required interface, can be placed on the entire component or on the operations and the implementation of the component shall respect this budget. Despite a sequential nature of the code, a component implementation may set specific non-functional constraints to preserve the functional correctness of its behaviour. Component implementation is thus a particularly attractive unit to be subcontracted to a third-party software supplier because the software architect can define components, attach technical budgets to it and delegate the implementation to he software suppliers. The software suppliers might add additional operations to the component implementation as and when necessary for the implementation \cite{CompBasedProcess}. 

\cref{fig: Component implementation} depicts one of the many possible component implementations for the component type \texttt{AOCS}

\begin{figure}[h]
	\centering
	\includegraphics[width=0.8\textwidth]{ComponentImplementation.pdf}
	\caption{Component Implementation}
	\source{\cite{CompBasedProcess}}
	\label{fig: Component implementation}
\end{figure} 

\item [Step 5: Definition of component instances] A component instance is an instance of a component implementation. It is a deployment unit which is subjected to allocation on a processing unit and it is an entity on which the non-functional properties are specified. Specifically, the non-functional properties are attached, as in \cref{fig: fig: Component instances}, to the provided interface side of the component, as they are the expression of a property or a provision of the component instance.

\item [Step 6: Definition of component bindings] Component bindings, as the name suggests, are the connections between one required interface of a component and the provided interface of another component. These bindings are set at design time and is subjected to static type matching to ensure that correct required and provided interfaces are connected to one another. This can be done by asserting the compatibility of the two interfaces (by type system or by inspection of the signature of their operations). If the binding is legal then whenever a call is made to an operation in the required interface, the call is dispatched to the correct operation in the bound provided interface. The signature of the calling operation in the RI (required interface) and the called operation in the PI (provided interface) are different and the connector, connecting these two interfaces, is in charge of performing this step. A tool support (possibly a back-end code generator) should initiate the configuration of the connector to perform this kind of binding.

It is also possible in this step to define bindings between an event emitter port of one component and an event receiver port of another component as shown in \cref{fig: Component instances}.

\item [Step 7: Specification of non-functional attributes] After component instances and component bindings have been defined, the software architect adds non-functional attributes to the services of the provided interfaces. 

In this step, the software architect specifies the timing and the synchronization attributes \cite{CompBasedProcess}. At first, the concurrency kind of the operation is established, and they can be \texttt{synchronous} or \texttt{asynchronous} operations. In case of a \texttt{synchronous} operation, it is executed in the flow of control of the caller and in case of an \texttt{asynchronous} operation, the operation is executed by a dedicated flow of control on the side of the callee. 

An \texttt{immediate} operation is said to be \texttt{protected} if it needs to be protected from data races in case of concurrent calls. The operation is said to be \texttt{unprotected} if it is free from such risks. In case of a \texttt{deferred} operation type, the architect can choose one of the following release patterns for the operation:

\begin{description}
\item [Periodic operation] The execution platform executes the operation at fixed periods with a dedicated flow of control.
\item [Sporadic operation] Two subsequent execution requests are separated by a minimum timespan called the minimum inter-arrival time (MIAT). The execution platform and the infrastructural code should guarantee this MIAT separation between two subsequent calls to the operation and the component implementer does not have to worry about it.
\item [Bursty operation] Only particular number of activations of an operation is allowed in a bounded interval of time. Again the execution platform and the infrastructure code guarantees this and the component implementer does not have to worry about it, as in the case for sporadic operation.
\end{description} 

For all the operations which have concurrency kind set as \texttt{asynchronous}, the software architect must provide the worst case execution time (WCET) of the operation. A preliminary value of WCET is initially provided based on previous use of operations in other projects (if any) and they can be refined with bounds at later stages after performing a timing analysis for a given target platform.

\cref{fig: Component instance} depicts the component bindings between the required and provided interfaces of the \texttt{AOCS} and \texttt{Mode\_Manager} component instances. It also depicts that non-functional properties which are specified on the services provided by the provided interfaces.

\begin{figure}[h]
	\centering
	\includegraphics[width=0.8\textwidth]{ComponentInstances.pdf}
	\caption{Component instance, component bindings and decoration with non-functional attributes}
	\source{\cite{CompBasedProcess}}
	\label{fig: Component instances}
\end{figure}

\item [Step 8: Definition of physical architecture] The hardware topology provides a description of the system hardware limited to the aspects related to communication, analysis and code generation. It also provides a model-level description of the relevant hardware of the system. In the hardware topology, following elements are described:
\begin{itemize}
\item Processing units that have a general-purpose processing capability
\item Avionics Equipment/Instruments/Remote terminals
\item The interconnection between the elements mentioned above 
\item A representation of the ground segment/other satellites (eg. Formation flying) to state the connection between the satellite and ground segment or other space segments
\end{itemize}		
For the specification of these elements, following attributes are used:
\begin{description}
\item [Processor frequency] This is used for processors to re-scale WCET values expressed in processor cycles in Step 6
\item [Bandwidth] This is used for buses and point-to-point links and it indicates maximum blocking time due to non-preemptability of the lower priority message transmission (for whatever reason), minimum and maximum size of packets, minimum and maximum propagation delay, the maximum time that the bus arbiter/driver needs to prepare and send a message on the physical channel and maximum time for the message to reach the receiver    
\end{description}

\item [Step 9: Component instances and component bindings deployment] In this step, the component instances are allocated on the processing units defined in the hardware topology (refer Step 8). In majority of the cases, it is straight-forward to allocate the bindings between the components as they are deployed on the same processing units \cite{CompBasedProcess}. In other cases, they need to be specifically allocated.

\item [Step 10: Model-based analysis] The system model developed within the software reference architecture is subjected to schedulability analysis to determine whether the timing requirements set in the interfaces can be met.

From the user model which is a PIM (Platform Independent Model), a Schedulability Analysis Model (SAM) which is a PSM (Platform Specific Model) is created. This model is subjected to analysis and the results of the analysis is available for the software architect as a read-only result. 

The analysis transformation chain requires a model representation of the generated containers and connectors to be defined in the SAM for an accurate analysis \cite{CompBasedProcess}.

Please note that, step 10 is not of concern in this Master thesis as this Master thesis deals only with automatic generation of containers and connectors and hence an accurate model based schedulability analysis is outside the scope. It is assumed in this Master thesis, that the user model successfully passes the Model based schedulability analysis and hence it is subjected to the model-code transformation directly. The actual flow is as depicted in \cref{fig: Automatic code generation}. Steps 8-9 are not of concern in this Master thesis, as it deals with hardware modeling and they are again outside the current scope of this Master thesis. However, these steps were mentioned for the sake of clarity and continuity.

\begin{figure}[h]
	\centering
	\includegraphics[width=0.4\textwidth]{AutomaticCodeGeneration.pdf}
	\caption{Generation of SAM and model-code transformation}
	\source{\cite{CompBasedDev}}
	\label{fig: Automatic code generation}
\end{figure}

\item [Step 11: Generation of containers and connectors]  This step is one of the main focus points of this Master thesis as mentioned before. Containers and connectors are generated and they specify:
\begin{itemize}
\item The structure of each container in terms of the required and provided interfaces of the enclosed component that they delegate and subsume
\item The structure of each connector 
\end{itemize} 
The non-functional attributes, component instances deployment and component connectors deployment play a major role in determining the creation of connectors and containers and how component instances and their operations are allocated to them.

Concurrency can be achieved by encapsulating sequential procedures into tasks which reside in containers and the protection from concurrent accesses can be provided by attaching them concurrency control structures. All of this can be achieved without modifying he sequential code and simply by following the use relations among the components.

In order for the OBSW to interact with the external world, sensors and actuators need to be provided. These hardware entities are represented as pseudo components (A pseudo-component indicates that a component is for interaction purposes only) and software capability is attached to these components at the component instance level.    
\end{description}

\cref{fig: Containers} depicts the automatic generation of containers and connectors for the components \texttt{AOCS} and \texttt{Mode\_Manager}. It also shows how their component instances are allocated to them. 

\begin{figure}[h]
	\centering
	\includegraphics[width=1.0\textwidth]{Containers.pdf}
	\caption{Automated generation of containers and connectors}
	\source{\cite{CompBasedProcess}}
	\label{fig: Containers}
\end{figure}

\section{Design flow and design views}
\label{section: Design flow and views}
When the component model is defined, it also defines implicitly a design flow as shown in \cref{fig: Design flow}, that needs to be followed, to be able to create an OBSW that meet all its user needs and high level requirements \cite{SAVOIR, PhdThesis, CompBasedProcess}. The design flow is as explained in the previous section. 

\begin{figure}[h]
	\centering
	\includegraphics[width=0.8\textwidth]{OverallFlow.pdf}
	\caption{The overall design flow}
	\label{fig: Design flow}
\end{figure}

The Obeo Designer Framework provides a concept called 'Viewpoint' and using this concept, the design views are implemented \cite{CompBasedProcess}. One of the advantages of the design views is to promote or enforce a certain design flow \cite{CompBasedProcess}. The component model is accompanied by the following design views:
\begin{description}
\item [Data view] This view is for the description of data types and events
\item [Component view] For definition of interfaces, components and the binding between them to fulfill their required needs
\item [Hardware view] For the specification of the hardware and the network topology
\item [Deployment view] For the allocation of components to computational nodes
\item [Non-functional view] In this view, the non-functional attributes are attached to the functional description of components
\item [Space-specific view] In this view, the services related to the commandability and observability of the spacecraft are specified
\end{description}  

\section{Language units of the OSRA}
\label{Language units}
The modeling language provided to the software architect to model the OBSW is divided for the ease of construction of the OBSW models into a set of language units \cite{SpecMetamodel}. Each language unit consists of closely related metamodel entities. The language units are grouped into separate meta-models for the sake of re-use as shown in \cref{fig: Language units} \cite{SpecMetamodel}. OSRA Component model is composed of the following language units:

\begin{figure}[h]
	\centering
	\includegraphics[width=0.4\textwidth]{LanguageUnits.pdf}
	\caption{Implementation of OSRA component model in the reference implementation}
	\source{\cite{CompBasedProcess}}
	\label{fig: Language units}
\end{figure}

\begin{description}
\item [\texttt{CommonKernel}] Defines the basic entities that are used as the base elements of the language architecture
\item [\texttt{DataTypes}] Defines all the possible data and the data types that can be used in an OBSW model
\item [\texttt{SCM Kernel}] Defines the infrastructural part of the Space Component Model (SCM) which can be considered as the language to express all the concerns expressed by an OBSW model
\item [\texttt{Component}] Defines a complete set of interfacing features (interfaces, events, datasets), component types implementations and interface ports
\item [\texttt{Non-functional properties}] Defines the non-functional properties that can be applied to the modeling entities and a new language called the Value Specification Language (VSL) to specify values characterized by the measurement units
\item [\texttt{Deployment}] Defines instantiation and deployment entities such as component instances, connection between them and their deployment on the hardware architecture
\item [\texttt{Monitor and Control (M\&C)}] Defines the means to specify the technical properties related to M\&C that shall be provided in the OBSW model
\item [\texttt{Hardware execution platform}] Defines entities related to the execution platform, Time Space Partitioning (TSP) and the hardware architecture   
\end{description}

\section{OSRA SCM Model Editor}
\label{section: OSRA editor} 
The toolset that the software architect can use to build OBSW models is organized as a set of Eclipse features and Eclipse plugins. 

The toolset is available as
\begin{itemize}
\item A pre-installed Eclipse (Eclipse Neon) for Windows 64-bit  
\item An update site which consists of a set of static files which can be placed locally, on a web-server or on a file-server. 
\end{itemize}
In the latter case, the software architect would have to use the Eclipse Update Manager to install the plugins \cite{OSRAEditor}. 

\cref{fig: OSRA model editor} shows a screenshot of the OSRA SCM model editor.

\begin{figure}[h]
	\centering
	\includegraphics[width=0.8\textwidth]{OSRAModelEditor.pdf}
	\caption{Screenshot of the OSRA SCM model editor}
	\label{fig: OSRA model editor}
\end{figure}

In line with the design flow and design views explained in section \cref{section: Design flow and views}, different OSRA diagrams can be created with the help of OSRA SCM Model editor namely:
\begin{description}
\item [Interfaces, Events and Datasets diagram] This is the first diagram of the OSRA activity and allows to define the data types, events, data sets and the interfaces that would be used by the components in the Component Types diagram.
\item [Component Types diagram] This is the second diagram of the OSRA activity and it allows to define the component types, device types, execution platform service types, partition proxy types, required ports whose implementation would be used by the component instance diagram
\item [Component Instance diagram] This is the third diagram of the OSRA activity and it allows to define the component instances, device instances, execution platform service instances, partition proxy instances, provided interface slots, data receiver slots and event receiver slots.
\item [Hardware diagram] This is the last diagram of the OSRA activity and it allows to define the hardware elements such as processor boards, mass memory units, devices, buses.
\end{description}

There are also tables which are provided for diagram elements (if applicable) and they are usually found as tabs in pop-up window associated with the group that the element belongs to. Tables are usually classical tables where rows represents an element and each column represent a potentially computed property of the element. Rows can also contain sub-rows recursively which represent the sub-elements and the software architect can collapse or expand these sub-elements as desired. More information and details about install requirements and procedure, usage of the OSRA editor can be found in the reference \cite{OSRAEditor}.    
    

  
% !TeX spellcheck = en_US

\chapter{Tasking Framework}
% !TeX spellcheck = en_US

\chapter{A programming model for OSRA}
\label{chap: Progamming model}

\section{Introduction}
In the previous chapters we have seen model-driven software development approach that was centered on component-based techniques. Dijkstra's principle of separation of concerns was one of the cornerstone principles which was part of the software reference architecture and the component model proposed \cite{CompBasedProcess,EvoRAVCodeAr}. According to it, the user design space should be limited to the internals of the components, where only strictly sequential code can be used and the extra non-functional requirements are declaratively specified in the form of annotations on the component provided interfaces. This is already explained in detail in the Step 7 (Specification of non-functional attributes) in \cref{chap: Software development process} 

As discussed in the previous chapters, the reference software architecture is made up of a component model, a computational model, a programming model and a conforming execution platform. It is also clear that the component model should be statically bound to a computational model to formally define the computational entities and the rules which govern their usage.

The realization of extra functional properties or more precisely, the generation of the complete infrastructure code can be done in two steps:
\begin{itemize}
\item Automated generation of the non-functional code, code for handling concurrency and interaction requirements for communication between components and the skeletons for the components themselves
\item Automated generation of containers for components and the connectors between components 
\end{itemize}

A code generator needs to be developed for this purpose and the next few chapters would be concerned about realizing the above mentioned steps. As a result, the third-party software supplier can solely concentrate on implementing the functional code of the components. This is in line with the principle of separation of concerns, which is of very high interest.

The ASSERT project (Automated proof-based System and Software Engineering for Real-Time systems), was the first, large project which showed the feasibility of a development approach for high-integrity real-time systems centered on separation of concerns, correctness-by-construction and property preservation.

In the ASSERT project, which incorporated Model-driven-engineering approaches \cite{PhdThesis}, a modelling infrastructure named RCM was developed at the University of Padua \cite{ScheduAnaly}. This infrastructure included a graphical modeling language and an editor, a model validator and set of model transformations that were necessary to feed model-based analysis and code generation \cite{ScheduAnaly}. Ravenscar Computational Model (RP) was chosen as a computational model in this modelling infrastructure and RP directly emanated from the Ada Ravenscar Profile in language-neutral terms \cite{EvoRAVCodeAr}. RP basically did not allow any language constructs that were exposed to unbounded execution-time and non-determinism \cite{CharEvoRAVCodeAr,EvoRAVCodeAr,RAVCodeAr}. Certain RP-compliant code archetypes were developed to complete the formulation of a programming model in the ASSERT project, which adhered to the vision of principle of separation of concerns and amenable to code generation \cite{CharEvoRAVCodeAr}. The code archetypes used Ada run-time, which is more compact in foot-print and hence could fit the needs of typical embedded systems which were resource-constrained \cite{RAVCodeAr}. The archetypes developed in the ASSERT project, were based on the previous work on code generation from HRT-HOOD to Ada \cite{CharEvoRAVCodeAr,EvoRAVCodeAr}.

In the following Artemis JU CHESS project, which was an initiative from ESA in parallel to the development of the SCM \cite{PhdThesis,CompBasedProcess}, these code archetypes from the ASSERT were revised by adding certain features \cite{EvoRAVCodeAr}. The code archetypes in the CHESS project also targeted the Ravenscar Computational Model for the additional reason that the reduced tasking model used in the Ada Ravenscar Profile matched the semantic assumptions and communication model of real-time theory, the response-time analysis in particular \cite{CharEvoRAVCodeAr}. The code archetypes developed in the CHESS project \cite{EvoRAVCodeAr} are taken as reference for developing a programming model in this chapter, for further use. The code archetypes discussed in this Master thesis however target the Tasking framework which is the chosen computational model for this Master thesis. The reasons for choosing Tasking framework as a computational model is already explained in the end of the previous chapter.

\section{Structure of the code archetypes}
The code archetypes discussed in this Master thesis strive to attain as much separation as possible between the functional and extra-functional concerns. At the implementation level, functional/algorithmic code of a component is separated from the code that manages the realization of the extra-functional requirements like tasking, synchronization and different time-related aspects.

The library of sequential code, which may have as many cohesive operations as the software supplier wishes to include in a single executing component, is included in a closed structure. The mapping of this structure to the actual design entity of the infrastructural code is not of concern in this chapter. The sequential code in this structure is executed by a distinct flow of control of the system. The dedicated flow of control can be an active task, together with other tasking primitives from the Tasking framework (if the desired concurrency kind is \texttt{asynchronous}) or a simple synchronous method/operation invocation using the flow of control of the component requesting the service from outside (if the desired concurrency kind is \texttt{synchronous}). This leads to a combined effect that the component internals are completely hidden from outside, but the provided services invoked by the external clients are executed with the desired interaction semantics.

As multiple clients may independently require the range of services to be executed by one of the two desired flow of controls, it is necessary to safeguard these execution requests. Safeguarding of execution requests in case of synchronous service requests is implicit as these requests would have been raised in the respective flow of control of the component asking for the service, but the safeguarding of execution requests in case of asynchronous requests needs to be handled. The handling of these kind of requests, is explained in the next parts of this section.

Service requests can often lead to valid/invalid data that need to be sent back safely to the components which made the requests. The service requester also need to be informed about any exceptions that might arise due to any unexpected situations during the servicing of the requests. The mechanisms and semantics necessary for realizing these requirements are also explained in the next parts of this section. 

\subsection{Synchronous release patterns}
The archetypes for a synchronous release pattern are quite straight forward. When a request for a service is made with the desired concurrency kind specified as \texttt{synchronous}, the request is handled straight-away as a normal function/operation call in the flow of control of the service requester. The results (if any) from the service requests, and exceptions (if any) during the course of handling the service requests are returned back to the service requester using the same flow of control.

\subsubsection{\textbf{Protected}}
When the non-functional property set on the service, in the provided interface side of the component offering the service is \texttt{Protected}, it is necessary for the container that wraps around the component to safeguard this non-functional property. As the container is the entity that promotes the provided interface of the component, it intercepts the function/operation call from outside and provides exclusive access to the service implemented by the component. Semaphores provided by the Tasking framework is used for this purpose.

\subsubsection{\texttt{Unprotected}}
When the non-functional property set on the service, in the provided interface side of the component offering the service is \texttt{Unprotected}, the semantics of handling the service request is essentially the same as the way the protected operations are handled, except for the fact that obtaining and releasing of the semaphore for the operation is not anymore needed.

\subsection{Asynchronous release patterns}
The archetypes for an asynchronous release pattern are quite complicated when compared to the way the requests with synchronous release pattern. As the requests cannot be anymore handled in the flow of control of the service requester, tasks from Tasking framework along with other tasking primitives, which are independent threads of execution can be activated to cater to these requests on the provided interface side. 

The asynchronous service request is initially intercepted at the required interface port subsumed by the container of the component which makes the request. Here the data (if any) associated with the request is packaged and the packaged data is forwarded to the provided interface port, which is promoted by the container of component handling the request. along with the data associated with the service request, it is also important that the required interface port packages information about how to send back the results (if any) and exceptions (if any) to the service requester.

Each thread of control, having its own structure as explained below, is responsible for only one operation in the provided interface side of the component that handles requests. As the release patterns for requests are already decided statically and as these release patterns are not expected to change at run-time, the number of threads of control that will be necessary to handle the service requests will be known at compile-time.

This is very similar to the way asynchronous release patterns are handled in the code archetype listed in \cite{CharEvoRAVCodeAr,EvoRAVCodeAr} except for the fact that they do not consider service requests which might result in results or exceptions that need to be sent back to the service requester \cite{CharEvoRAVCodeAr}.  

\subsubsection{\textbf{Sporadic}}
When the non-functional property set for the handling of the service, on the provided interface side of the component offering the service is \texttt{Sporadic}, it is the responsibility of the container of the service provider component to safeguard this property. The sporadic property requires that two subsequent requests for the service needs to always be separated by no less but possibly more than a minimum guaranteed time span, known as the MIAT (Minimum Inter-Arrival Time) \cite{SpecMetamodel,CompBasedProcess}. The container makes use of tasking primitives such as a task channel, task event and a task from the Tasking framework for this purpose.

The general structure of the thread of control on the service provider end, necessary to handle sporadic service requests consists of a task with two synchronized task inputs, attached to it. One of the task inputs is associated with a task event, with absolute timing (fixed task wake-up times) and the other task input is associated with a normal task channel. The task event is configured to wake up the task periodically after every MIAT interval. The task input associated with a normal task channel is configured so that the task input is activated as soon as a push is made against its associated task channel. This task then is instantiated in the container of the service provider component.  

When a provided interface port, promoted by the container of the component handling the request, receives a sporadic service release request, it intercepts the request and pushes the packaged data against the channel associated with the task. 

Because of synchronized task inputs, the task is activated only after both its task inputs are activated. When activated, the functions of the task will then be to:
 
\begin{description}
\item [Step 1] Unpack the packaged data
\item [Step 2] Acquire the semaphore provided by the Tasking framework associated with the service
\item [Step 3] Execute the desired service
\item [Step 4] Reset the task event attached to the task
\item [Step 5] Release the semaphore acquired 
\item [Step 6] Return the results and the exceptions associated with the service request back to the service requester making use of the information of the service requester packaged by the required interface port 
\end{description}

In this way, the non-functional properties associated with an asynchronous sporadic release pattern can be preserved at run-time.

\subsubsection{\textbf{Protected}}
When the non-functional property set for the handling of the service, on the provided interface side of the component offering the service is \texttt{Protected}, it is the responsibility of the container of the service provider component to safeguard this property. The container makes use of tasking primitives such as a task channel and a task from the Tasking framework for this purpose. 

The general structure of the thread of control on the service provider end, necessary to handle this kind of service requests, is a task with one task input attached to it. The task input is configured so that the task input is activated as soon as a push is made against its associated task channel.  

When a provided interface port, promoted by the container of the component handling the request, receives a service release request of this kind, it intercepts the request and pushes the packaged data against the channel associated with the task. The task is then activated and the functions of the task will then be to:

\begin{description}
\item [Step 1] Unpack the packaged data
\item [Step 2] Acquire the semaphore provided by the Tasking framework associated with the service
\item [Step 3] Execute the desired service
\item [Step 4] Release the semaphore acquired 
\item [Step 5] Return the results and the exceptions associated with the service request back to the service requester making use of the information of the service requester packaged by the required interface port 
\end{description}

In this way, the non-functional properties associated with an asynchronous protected release pattern can be preserved at run-time.

\subsubsection{\textbf{Bursty}}
When the non-functional property set for the handling of the service, on the provided interface side of the component offering the service is \texttt{Bursty}, it is the responsibility of the container of the service provider component to safeguard this property. The bursty property requires that a service can be activated at most a given number of times in a given interval called the bound interval \cite{SpecMetamodel,CompBasedProcess}.

The general structure of the thread of control on the service provider end, necessary to handle this kind of service requests is a task with two non-synchronized task inputs attached to it. One of the task inputs is associated with a task event, with absolute timing (fixed task wake-up times) and the other task input is associated with a normal task channel. The task event is configured to wake up the task periodically after every bound interval. The task input associated with a normal task channel is configured in a way that the task input is activated as soon as a push is made against its associated task channel. The task also has an internal counting semaphore provided by the Tasking framework in order to keep a count of the number of service requests handled within the bound interval.

When a provided interface port, promoted by the container of the component handling the request, receives a service release request with bursty nature, it intercepts the request and pushes the packaged data against the channel associated with the task.
  
Because of non-synchronized task inputs, the task is activated if any one of its task inputs are activated. When activated, the functions of the task will then be to:

\begin{description}
\item [Step 1] Check the activated input. If the activated input is the one that is attached to a task event, then replenish the counting semaphore, restart the attached task event and go to step 8. 
\item [Step 2] Unpack the packaged data
\item [Step 3] Acquire the counting semaphore local to the task which is used to enforce the max. number of activations within a bound interval
\item [Step 4] Acquire the semaphore provided by the Tasking framework associated with the service
\item [Step 5] Execute the desired service
\item [Step 6] Release the semaphore associated with the service 
\item [Step 7]Return the results and the exceptions associated with the service request back to the service requester making use of the information of the service requester packaged by the required interface port 
\end{description}

In this way, the non-functional properties associated with an asynchronous bursty release pattern can be preserved at run-time. It is important to note that the code archetypes which were developed for the CHESS project do not mention the scheme to handle this kind of release pattern \cite{CharEvoRAVCodeAr,EvoRAVCodeAr}. It is unclear from the available resources whether, the reason to not consider this code archetype, was because of the non-possibility to opt this non-functional property for a service request on the provided interface side.

\subsubsection{\textbf{Cyclic}} 
When the non-functional property set for the handling of the service, on the provided interface side of the component offering the service is \texttt{Cyclic}, it is the responsibility of the container of the service provider component to safeguard this property. Cyclic property requires that the associated request be activated periodically and with a non-zero initial offset \cite{SpecMetamodel,CompBasedProcess}. 

The general structure of the thread of control on the service provider end, necessary to handle this kind of service requests is a task with a task event provided by the Tasking framework, attached to it. The task event is configured to wake up the task periodically, with absolute timing (fixed task wake-up times). The task event can also be configured to wake up the associated task for the very first time with an initial offset.

When a provided interface port, promoted by the container of the component has a service which needs to be activated periodically, the task is activated and it performs the following functions:

\begin{description}
\item [Step 1] Acquire the semaphore provided by the Tasking framework associated with the service
\item [Step 2] Execute the desired service
\item [Step 3] Release the semaphore acquired 
\end{description}    

The service with a cyclic nature cannot be requested from an external component \cite{SpecMetamodel}. The services also need to be parameter less and cannot send out results or throw exceptions \cite{SpecMetamodel}. 

In this way, the non-functional properties associated with an asynchronous cyclic release pattern can be preserved at run-time.   

   



 


 

% !TeX spellcheck = en_US

\chapter{Infrastructural code generation}
\label{chap: Code generation}
\section{Introduction}
After designing an OBSW model using the OSRA editor and following the component-based software development approach that comes it, the OBSW model entities need to be mapped to the infrastructure code. The reference programming model discussed in the previous chapter could be used with OSRA helps us in progressing towards this goal. But, it is necessary to understand the overall design approach for the generated code and briefly present the abstractions that will be offered to the software supplier and these topics are dealt with in detail in this chapter.

Similar efforts were are also performed in the Artemis JU CHESS project \cite{EvoRAVCodeAr} and these efforts provides the perfect base for discussions in this chapter of the Master thesis.   

\section{User model entities in the Platform Independent Model (PIM) phase}
A detailed description of all the modeling entities that the software architect can use, can be found in the specification of the metamodel for the OSRA component model \cite{SpecMetamodel}. However a brief description of them is useful here:
 
\begin{description}
\item [Datatypes] The software architect can create a set of project-specific data types and constants using the Datatypes language unit of the CommponTypes metamodel and the language unit is designed to provide the software architect an expressive power comparable to languages with strong types (e.g. Ada).\cite{SpecMetamodel}. The supported type definitions are boolean types, integer types, float types, enumeration types, fixed point types, array types, structured types, string types, union types, alias types, opaque types, external types and unconstrained types. Some of the datatype definitions are obvious for readers with programming skills in types languages such as Ada, C or C++. 

\item [Interfaces] An interface is a specification of coherent set of services and it represents the definition of a contract. An interface is defined independently of the entities implementing it (e.g. Component type). An interface may enlist declaration of operations, which are the functional services that shall be offered by the entities implementing it. The services include a name, set of ordered parameters. Parameters are typed with one of the types mentioned above and have a mode (in, out or inout). A component type may expose one or more interfaces and the same interface can be exposed by different component types. An interface may also contain the declaration of one or more interface attributes, which are the parameters that are accessible via the interface implementation.

\item [Component type] Component type is an entity which specifies the external interface of a software component and are defined in isolation and used to declare relationships with the other components and system in general. It conforms to the principle of encapsulation and a consequence all the interactions with other components are performed exclusively via its explicitly declared interface. Component type usually encompasses:
\begin{itemize}
\item List of provided interface ports
\item List of required interface ports
\item List of dataset emitter ports 
\item List of dataset receiver ports
\item List of event emitter ports
\item List of event receiver ports 
\end{itemize}

\item [Component implementation] It is the entity that represents a concrete realization of a component type. It is functionally identical to the component type and the source code is added to the component implementation and may also define number of component implementation attributes

\item [Component instance] It is an instantiation of a component implementation and hence contains all the instantiations of the structural features (Different ports). It also contains instantiation of all attributes (interface attributes, component type attributes and component implementation attributes). It is also the elementary deployment unit for the OSRA component model.        
\end{description}

\section{Mapping of design entities to the infrastructural code}
As the generated code should target the Tasking framework, which is the target computational model in this Master thesis and because the Tasking framework is written in C++, the following sections explains mapping of design entities to the infrastructure code that will be generated in C++. Certain terms specific to C++ only are used in this section.

On analyzing the specification of the metamodel for the OSRA component model \cite{SpecMetamodel}, it is clear that there might be different corner cases that might possibly arise in the construction of the OBSW models using OSRA component model and it is necessary that these corner cases are effectively handled in the software design for the infrastructural code. An effort is made to first of all build an OBSW model which incorporates these corner cases and in the following sections, an attempt is made to answer the questions at hand.

\subsection{An example OBSW model}
Our simple OBSW model, yet effective to serve the intended purpose, is built as per the proposed component-based development approach explained in the section \cref{Design steps} in chapter \cref{Software development process}. As already mentioned in that section the component-based approach puts a lot of emphasis on the definition of component interfaces \cite{CompBasedProcess} and it is followed here as well. Components are built from scratch using newly defined interfaces

\begin{description}
\item [Step 1: Definition of data types and events] As the Master thesis requires to emphasize more on capturing interaction and concurrency semantics required for communication between the designed components effectively, the data types chosen in this example are fairly simple. But it is important to note that the scheme of mapping of these simple datatypes to the infrastructural code, can be scaled to fairly complex datatypes as well.

Two datatypes namely \texttt{FixedLengthStringType} and \texttt{IntegerType} with \texttt{integerKind} set to \texttt{UNSIGNED} are defined (instantiated from the OSRA Component Model \cite{SpecMetamodel}) and they are named as \texttt{StringType} and \texttt{IntegerType} respectively. Three exception types, named as \texttt{OperandException}, \texttt{MemoryException} and \texttt{OverflowException} are defined. An \texttt{Event} type, which can be used for asynchronous notification \cite{SpecMetamodel} is instantiated and it is named as \texttt{FailureEvent}. Two parameters namely \texttt{m\_Param} and \texttt{m\_Description} with datatypes \texttt{IntegerType} and \texttt{StringType} are instantiated as parameters of the event \texttt{FailureEvent}

\item [Step 2: Definition of interfaces] Two interface namely \texttt{InterfaceA} and \texttt{InterfaceB} are designed. \texttt{InterfaceA} has only one single operation by name \texttt{callOperationAdd} and \texttt{InterfaceB} has an operation by name \texttt{OperationAdd} and an interface attribute of datatype \texttt{IntegerType} named as \texttt{m\_StatusValue}.

The \texttt{OperationAdd} has three parameters, out of which two parameters have \texttt{ParameterDirectionKind} set to \texttt{in} and the third parameter has the \texttt{ParameterDirectionKind} set to \texttt{out}. The \texttt{OperationAdd} is also configured to return any of the three exception kinds mentioned in the previous step. The interface attribute \texttt{m\_StatusValue} has the \texttt{AttributeKind} set to \texttt{CFG} which indicates that the interface attribute parameter is a configurable parameter \texttt{SpecMetamodel}  

\item [Step 3: Component types] Component types namely \texttt{Component\_Caller} and \texttt{Component\_Callee} which form the basis for a reusable software asset are defined. 

\texttt{Component\_Caller} has one provided interface port named as \texttt{ProvidedInterfacePort} and two required interface ports named as \texttt{RequiredInterfacePortType1} and \texttt{RequiredInterfacePortType2}. \texttt{ProvidedInterfacePort} refers to \texttt{InterfaceA} and both \texttt{RequiredInterfacePortType1} and \texttt{RequiredInterfacePortType2} refer to \texttt{InterfaceB}. All the operations in the \texttt{RequiredInterfacePortType1} have the desired interaction kind set to \texttt{synchronous} and all the operations in the \texttt{RequiredInterfacePortType2} have the desired interaction kind set to \texttt{asynchronous} (Note that, it is possible to independently choose the desired interaction kind for each operation \cite{SpecMetamodel}).

\texttt{Component\_Callee} has two provided interface ports named as \texttt{ProvidedInterfacePort1} and \texttt{ProvidedInterfacePort2} and no required interface port. Both \texttt{ProvidedInterfacePort1} and \texttt{ProvidedInterfacePort2} refer to \texttt{InterfaceB}.

\item [Step 4: Component implementation]

\end{description}
   
\subsection{Interface}
An interface can be mapped as a pure virtual base class in C++. For each operation in the interface, a function is added to this virtual base class. The possible parameters of the operation are also added   

\section{Mapping of design entities to the infrastructural code}    
In this section, it is explained how the design entities could be mapped to the infrastructure code that needs to be generated with the model-code transformations through the help of a simple example:

\subsection{Problem description}
A very simple OBSW model consists of two components, namely \texttt{ComponentA} and \texttt{ComponentB} 

\texttt{ComponentA} has the following requirements:
\begin{itemize}
\item The function \texttt{startOperation} implemented in the component implementation needs to be called on the periodic basis with a period of two seconds.

\item There should be two required interface ports. 
\begin{itemize}
\item First required interface port allows \texttt{ComponentA} to call an operation named \texttt{operationAdd}, set and get the status value \texttt{statusValue} with concurrency kind \texttt{immediate}
\item Second required interface port allows \texttt{ComponentA} to call the same operation \texttt{operationAdd}, allows to set and get the same status value \texttt{statusValue} with concurrency kind as \texttt{deffered}. 
\end{itemize}

\item Can receive messages or events asynchronously and there should be a event receiver port which listens to a particular event called the \texttt{FailureEvent} 

\item Can receive a report \texttt{OperationAddReport} for the operation \texttt{operationAdd} which is called. The report has status of the operation \texttt{operationAdd} called the \texttt{operationAddStatus} and the exception for the operation \texttt{operationAdd} called the \texttt{OperationAddException} 
\end{itemize}

\texttt{ComponentB} has the following requirements:
\begin{itemize}
\item Provides two component implementations. Both the implementations provides their own
\begin{itemize}
\item Versions of the operation \texttt{operationAdd} which can be called by any other components
\item Instances of the status value \texttt{statusValue} which can be written to and read from other components 
\end{itemize}

\item Two provided interface ports and they have non-functional requirements attached to them 
\begin{itemize}
\item First provided interface port works with the first version of component implementation and it requires \texttt{operationAdd} to be a \texttt{protected} operation and needs the status value \texttt{statusValue} to be set and get in an unprotected way
\item Second provided interface port works with the second version of component implementation and it requires \texttt{operationAdd} to be called in a sporadic way with a \texttt{Minimum Inter-Arrival Time (MIAT)} of two seconds and needs the status value \texttt{statusValue} to be set and get in an protected way.
\end{itemize}

\item Can send messages or events asynchronously and there should be a event emitter port which emits a particular event called the \texttt{FailureEvent}

\item Can send report report \texttt{OperationAddReport} for the operation \texttt{operationAdd} which is called. The report has status of the operation \texttt{operationAdd} called the \texttt{operationAddStatus} and the exception for the operation \texttt{operationAdd} called the \texttt{OperationAddException} 
\end{itemize}  

From the above description it is clear that these components can be connected to each other or in particular the first required interface port can be connected to the first provided interface port and the second required interface port can be connected to the second provided interface port. The following different design entities are thereby constructed:

\subsection{Design entities}
\subsubsection{Interfaces}
Interfaces are implemented as C++ pure virtual classes and they have entries for the interface operations, interface attribute access operations and interface attributes. Concrete implementations for the interface operations and the interface attribute access operations need to be provided by the classes that implement these interfaces

For the above example:
 
Two primary interfaces can be created based on the problem description namely: 
\begin{description}
\item [\texttt{InterfaceA}] This specifies a single operation \texttt{startOperation}
\item [\texttt{InterfaceB}] This specifies 
\begin{itemize}
\item The operation \texttt{operationAdd}
\item Actual status value \texttt{statusValue}
\item Getter operation for the status value \texttt{statusValue} called \texttt{getStatusValue}
\item Setter operation for the status value \texttt{statusValue} called \texttt{setStatusValue}
\end{itemize}
\end{description}

Three secondary interfaces for \texttt{InterfaceB} are created: 
\begin{description}
\item [\texttt{InterfaceB\textunderscore Synchronous}] which redefines the interface \texttt{InterfaceB} for cases where the operations in the interface \texttt{InterfaceB} are called synchronously
\item [\texttt{InterfaceB\textunderscore Asynchronous}] which redefines the interface \texttt{InterfaceB} for cases where the operations in the interface \texttt{InterfaceB} are called asynchronously
\item [\texttt{InterfaceB\textunderscore extension}] which inherits the interface \texttt{InterfaceB} to add additional operations to handle the situation when the operations in the interface \texttt{InterfaceB} are called asynchronously
\end{description}

\subsubsection{Operation parameter structures}
As the operations can be called with concurrency kind defined as \texttt{deferred}, it is necessary to pack in C++ structures the parameters of the operation (if any) and the address of the component to which the result of the operation (if any), status report of the operation (if any) needs to be sent. This encapsulates all the data necessary to compute the operation, to store the result of the operation and the location to which the result of the operation needs to be sent.  

For the above example:

Two operation parameter structures are required namely:
\begin{description}
\item [\texttt{operationAddStruct}] which holds the parameters of the operation \texttt{operationAdd} and also the address of the component to which the result of the operation \texttt{operationAdd} needs to be sent back
\item [\texttt{statusValueStruct}] which holds the status value and also the address of the component to which the result of the operation \texttt{getStatusValue} needs to be sent
\end{description}

\subsubsection{Operation exceptions}
As the operations executed can lead to different kind of exceptions, they must be delivered back to the component which called the operation. They can be defined as enums in C++.

For the above example:

As the operation \texttt{operationAdd} can raise an exception \texttt{OperationAddException}, it is defined as an enum with enum parameters \texttt{OperandException, MemoryException, OverflowException, none}

\subsubsection{Operation status}
The execution status of the operations may be necessary to be delivered to the caller and the statuses are defined as an enums in C++.

For the above example:

The status of the operation \texttt{operationAdd} can be sent to the component which makes the call. It is defined as an enum \texttt{OperationAddStatus} with enum parameters \texttt{Started, Running, Finished}  

\subsubsection{Operation reports}
For a particular operation, the enums of exceptions and status descriptions can be instantiated in a report. These reports are realized as C++ structs.

For the above example:

One operation report is required namely:
\begin{description}
\item [\texttt{OperationAddReport}] holds the instantiation of the enum \texttt{OperationAddException} and of the enum \texttt{OperationAddStatus} 
\end{description}     

\subsubsection{Call back operations and event receptions}
Pure virtual C++ classes are needed for specifying:
\begin{itemize}
\item The call back operations for 
\begin{itemize}
\item Results of the operations (if any)
\item The statuses of the operations (if any)
\end{itemize}
\item The call back operations for interface attribute getter operations
\item Event reception operations
\end{itemize}

For the above example:

Three pure virtual classes are required namely:
\begin{description}
\item [\texttt{AsynchronousRequirementsOperationAdd}] specifies the call back operation for operation \texttt{operationAdd} which consist of the result of the operation available in operation parameter structure \texttt{OperationAddStruct} and the report for the operation \texttt{OperationAddReport}
\item [\texttt{AsynchronousRequirementsStatusValue}] specifies the call back operation for getter operation of the status \texttt{statusValue}
\item [\texttt{AsynchronousRequirementsFailureEventReception}] specifies the event reception operation 
\end{description}

\subsubsection{Component types}
Component types are implemented as pure virtual classes in C++. A component type specifies the means for instances of it to connect with other components. They realize the interfaces and also realize the pure virtual classes which specify the call back operations and the event receptions.

For the above example:

There are two component types namely:
\begin{description}
\item [\texttt{ComponentType\textunderscore Caller}] implements three pure virtual classes namely:
\begin{itemize}
\item \texttt{InterfaceA}
\item \texttt{AsynchronousRequirementsOperationAdd} as it calls operation \texttt{operationAdd} with concurrency kind \texttt{deferred}
\item \texttt{AsynchronousRequirementsStatusValue} as it calls operation \texttt{getStatusValue} with concurrency kind \texttt{deferred}
\item \texttt{AsynchronousRequirementsFailureEventReception} as it receives event \texttt{FailureEvent}
\end{itemize}
\item [\texttt{ComponentType\textunderscore Callee}] implements \texttt{InterfaceB} only
\end{description}

\subsubsection{Component implementations}
Component implementations contain the definition of the component type and contains concrete implementations for all the operations in the interfaces it inherits from, for all the call back operations and the event reception operations which were specified in their respective classes. Component implementations are implemented as instantiable C++ classes.

For the above example:

There are two component implementations namely:
\begin{description}
\item [\texttt{ComponentImplementation\textunderscore Caller}] inherits from the \texttt{ComponentType\textunderscore Caller} and provides concrete implementations for all the classes that it indirectly inherits from
\item [\texttt{ComponentImplementation\textunderscore Callee}] A    

   



 


 

% !TeX spellcheck = en_US

\chapter{Evaluation of the code generator}
\label{chap: code generator evaluation}
\section{Introduction}
By using model-driven engineering tools (MDE) tools for code generation, it is possible to generate software code automatically and achieve extremely high developer productivity rates of thousands of function points and millions of lines of code per person-month \cite{EvalCodeGen}. But, as we have seen in the previous chapters, the MDE approach consists of more than code generation tools; It defines the entire software-engineering approach that can impact the entire lifecycle from requirements gathering through sustainment \cite{CompBasedProcess, SAVOIR}.     

It is important to consider these tools and methods in the context of a particular system acquisition i.e., the MDE methods and tools need to be aligned with the system acquisition strategies, which would in turn improve system quality, reduce time to field, and reduce sustainment cost \cite{EvalCodeGen}. System acquisition strategies include: 

\begin{itemize}
\item Securing the necessary data rights and licensing for tools, models, generated code, run-time libraries, frameworks, and other supporting software
\item Reviewing and evaluating appropriate artifacts introduced by the MDE tools at the right time in the acquisition cycle
\item Approaches to manage program risks, include risk identification and mitigation 
\end{itemize}

If the methods and tools do not align with the system acquisition strategy, using them can result in increased risk and cost in development and sustainment. The acquirers in government or large commercial enterprises have the challenge of selecting contractors to develop their systems. The tools and processes selected by the contractors and developers have direct impact on the software quality concerns of the acquirer, who often has little influence on the selection of these tools and processes. The tool acquirer would then have to answer the following acquisition evaluation questions \cite{EvalCodeGen}:

\begin{itemize}
\item Do the engineering processes and associated development tools match the desired acquisition strategy?
\item Do the tools support the developer's software development methodology?
\item Are the code generation tools capable of integrating with other development and management tools to support measurement and monitoring of the development progress?
\item Will the selected development methodology with its associated tools be available and compatible for the expected lifecycle of the system? 
\end{itemize}

This chapter subjects the code generator developed as a part of the Master thesis to evaluation methods listed in \cite{EvalCodeGen} and provide necessary inputs for conducting the acquisition evaluation.

\section{Selection and evaluation methods of a MDE tool for code generation}
The step-by-step MDE tool selection process defined in \cite{EvalCodeGen} makes use of the PECA method \cite{PECAProcess} as shown in \cref{fig: PECA method}.  

As a part of the \texttt{Establish criteria} in the PECA process, the acquirer must establish criteria with which he has to decide whether a particular tool for automatic code generation is suitable for a specific system acquisition. Such a criteria can be developed using risk taxonomy \cite{RiskTax} which ensures that all relevant acquisitions strategies are covered i.e., it provides a checklist to ensure all potential risks are considered \cite{EvalCodeGen}. The risk taxonomy has three main sections \cite{RiskTax}:

\begin{description}
\item [Product engineering] This covers activities that create a system that satisfies the specified requirements and customer expectations. Risks in this area generally arise from requirements that arise from requirements that are technically difficult to achieve, inadequate requirements and design analysis, or poor design and implementation quality 
\item [Development environment] This includes risks related to the development process and system, management methods, and work environment
\item [Program constraints] This cover risks that arise from factors external to the project 
\end{description}

To establish a criteria for a particular program/project, it is necessary for the program to first scan the risk taxonomy and identify those areas that apply for the project. Each risk creates one or more acquisition concerns , which may refine the program risk or indicate how certain tool features or capabilities might help mitigate the risk \cite{EvalCodeGen}.      

As part of the \texttt{Collect data} step in the PECA process, a vendor self-assessment questionnaire is prepared as in \cite{EvalCodeGen} which is given to the MDE tool vendors, to provide data needed to make the tool selection decision.

% Please add the following required packages to your document preamble:
% \usepackage{multirow}
% \usepackage{graphicx}
\begin{table}[]
	\centering
	\caption{My caption}
	\label{my-label}
	\resizebox{\textwidth}{!}{%
		\begin{tabular}{lll}
			\hline
			\multicolumn{1}{|c|}{\textbf{Risk area}} & \multicolumn{1}{c|}{\textbf{\begin{tabular}[c]{@{}c@{}}Potential Acquisition \\ Concerns \\ Related to MDE Tools for \\ Automatic Code Generation\end{tabular}}} & \multicolumn{1}{c|}{\textbf{\begin{tabular}[c]{@{}c@{}}Answers to the linked questions \\ in the questionnaire\end{tabular}}} \\ \hline
			\multicolumn{3}{|l|}{\textbf{Requirements}} \\ \hline
			\multicolumn{1}{|l|}{\multirow{3}{*}{Stability}} & \multicolumn{1}{l|}{\begin{tabular}[c]{@{}l@{}}Responding to requirements \\ changes may necessitate \\ operating on partially complete \\ models and performing \\ refactoring or rework on \\ models.\end{tabular}} & \multicolumn{1}{l|}{\begin{tabular}[c]{@{}l@{}}The OBSW models designed using the OSRA Component \\ Model should be subjected to model validation against the \\ OSRA Specification Compliance and the SCM Metamodel \\ Compliance before it is subjected to automatic code \\ generation. In that case, the OBSW model, even if partially \\ complete can be subjected to code generation if it clears the \\ model validation step.\end{tabular}} \\ \cline{2-3} 
			\multicolumn{1}{|l|}{} & \multicolumn{1}{l|}{\begin{tabular}[c]{@{}l@{}}Communication with stake-\\ holders is partially important \\ to resolve requirements issues, \\ so tool features that support \\ this become more important\end{tabular}} & \multicolumn{1}{l|}{\begin{tabular}[c]{@{}l@{}}It is possible to annotate each of the model entities while \\ constructing the OBSW model with information which \\ can be used for communicating with the stakeholders. \\ There is no additional documentation too which comes with \\ the OSRA SCM tool suite\end{tabular}} \\ \cline{2-3} 
			\multicolumn{1}{|l|}{} & \multicolumn{1}{l|}{\begin{tabular}[c]{@{}l@{}}Interfaces between the soft-\\ ware modeling tools and \\ the requirements management \\ tools promote co-evolution \\ of requirements and software\end{tabular}} & \multicolumn{1}{l|}{\begin{tabular}[c]{@{}l@{}}There is no support for tracing requirements into model \\ elements at the current state of development of OSRA\\ SCM\end{tabular}} \\ \hline
			\multicolumn{1}{|l|}{\begin{tabular}[c]{@{}l@{}}Completeness\\ Clarity\\ Validity\end{tabular}} & \multicolumn{1}{l|}{\begin{tabular}[c]{@{}l@{}}In addition to the concerns \\ noted above about require-\\ ments stability, the ability to \\ execute or simulate the exe-\\ cution of the model can help\\ validate requirements \\ completeness\end{tabular}} & \multicolumn{1}{l|}{\begin{tabular}[c]{@{}l@{}}At the current stage of the development of the OSRA \\ SCM, it is not possible visualize the execution of the model. \\ It is only possible to create static models of the OBSW \\ and it is not possible to trace the flow of execution \\ through the model, or inject data or events into the model\end{tabular}} \\ \hline
			\multicolumn{1}{|l|}{Feasibility} & \multicolumn{1}{l|}{\begin{tabular}[c]{@{}l@{}}The ability to perform analysis \\ of the model for qualities \\ such as latency, throughput \\ and consistency can help \\ demonstrate the feasibility \\ of requirements\end{tabular}} & \multicolumn{1}{l|}{\begin{tabular}[c]{@{}l@{}}The model-based static analysis of the OBSW model is not \\ possible at the current stage of development of the OSRA. \\ Step 10 of the overall software development process \\ in cref\{section: Design steps\} in chapter cref\{chap:\\ Software development process\} gives an idea about the \\ analysis of latency, throughput etc. which can be perform-\\ ed on the OBSW model in the future\end{tabular}} \\ \hline
			\multicolumn{1}{|l|}{Scale} & \multicolumn{1}{l|}{\begin{tabular}[c]{@{}l@{}}Limitation on the size or \\ complexity of the model \\ that can be represented,\\ analyzed, or transformed \\ by the tool will limit the scale \\ of the system that can be \\ created\end{tabular}} & \multicolumn{1}{l|}{\begin{tabular}[c]{@{}l@{}}There are no size and complexity limitations for representing \\ OBSW models using the OSRA SCM tools\end{tabular}} \\ \hline
			&  &  \\
			&  &  \\
			&  &  \\
			&  &  \\
			&  & 
		\end{tabular}%
	}
\end{table}


% Please add the following required packages to your document preamble:
% \usepackage{graphicx}
\begin{table}[]
	\centering
	\caption{My caption}
	\label{my-label}
	\resizebox{\textwidth}{!}{%
		\begin{tabular}{lll}
			\hline
			\multicolumn{1}{|c|}{\textbf{Risk area}} & \multicolumn{1}{c|}{\textbf{\begin{tabular}[c]{@{}c@{}}Potential Acquisition \\ Concerns \\ Related to MDE Tools for \\ Automatic Code Generation\end{tabular}}} & \multicolumn{1}{c|}{\textbf{\begin{tabular}[c]{@{}c@{}}Answers to the linked questions \\ in the questionnaire\end{tabular}}} \\ \hline
			\multicolumn{3}{|l|}{\textbf{Design}} \\ \hline
			\multicolumn{1}{|l|}{Interfaces} & \multicolumn{1}{l|}{\begin{tabular}[c]{@{}l@{}}If only parts of the system \\ will be automatically gene-\\ rated, while other parts will \\ be developed using traditional \\ approaches, the interfaces bet-\\ ween these two types of \\ software must be designed \\ and developed\end{tabular}} & \multicolumn{1}{l|}{\begin{tabular}[c]{@{}l@{}}In the software design for the generated infrastructure code, \\ the model entities are mapped to infrastructural code\\ entities. As the model entities clearly separates the two \\ types of code, the generated infrastructural code entities\\ as well clearly separates the generated code from the user \\ code, which may be developed using traditional approaches. \\ The generated source code needs to be compiled. The \\ generated source code is in C++ and works with the \\ Tasking framework written in C++. The  code is generated\\ for the Linux platform and GNU GCC compiler can be \\ used to compile to the source code\end{tabular}} \\ \hline
			\multicolumn{1}{|l|}{Testability} & \multicolumn{1}{l|}{\begin{tabular}[c]{@{}l@{}}The tool should generate code\\ that exposes internal states and \\ interfaces needed to test the\\ generated software\end{tabular}} & \multicolumn{1}{l|}{\begin{tabular}[c]{@{}l@{}}The generated infrastructural code is testable as testability\\ of the generated code is one of the main concerns in the\\ software design for the infrastructural code\end{tabular}} \\ \hline
			&  &  \\
			&  &  \\
			&  &  \\
			&  &  \\
			&  &  \\ \hline
		\end{tabular}%
	}
\end{table}
% !TeX spellcheck = en_US

\chapter{Results and Conclusions}
\label{chap:conclusion}

\section{Discussion}
As a part of this Master thesis, 
\begin{itemize}
\item A choice is made to use the Tasking framework as a computational model so that the OSRA component model statically binds to Tasking framework which formally defines the computational entities and the rules which govern their usage 
\item A reference programming model is decided upon that enforces the analysis assumptions and which permits to express exclusively the semantics imposed by the analysis theory and which conveys the implementations of the desired non-functional properties using the primitives from the Tasking framework
\item Different corner cases which might arise during the construction of an OBSW model using the OSRA component model are identified 
\item An overall software design approach for the generated infrastructure code of the OBSW models is presented and a mapping of the OBSW model design entities to the infrastructural code entities is presented. The generated code would then have all the good characteristics of a software as listed in \cref{subsection: Software design approach} 
\item A code generator is implemented, using which the generation of the entire non-functional code i.e., the code for handling the concurrency and interaction requirements for communication between components and generation of component containers and component connectors can be automated. The code generator uses the already tried and tested Tasking framework as the platform and bases the generated code on it. The advantage of this is that it eases the model-to-code transformation step 
\item The implemented code generator is tested for multiple OBSW models as shown in \cref{chap: Extra examples} which capture the different corner cases identified
\item For the simple OBSW model example which was introduced in \cref{chap: Code generation}, a set of unit test cases are written using Gtest and Gmock frameworks and the test coverage reports are generated
\end{itemize}

The following results were obtained 
\begin{itemize}
\item  The implemented code generator successfully generates the infrastructural code entities for all the example OBSW models listed in \cref{chap: Code generation} and in \cref{chap: Extra examples}. The generated code in all cases is successfully compiled along with Tasking framework using GCC C++ compiler conforming to the C++11 standard for the Linux platform
\item The test coverage reports generated for the unit tests written for the simple OBSW example are analyzed. The results show that the testability factor of the generated code is high and the infrastructural code entities can be efficiently tested      
\end{itemize}

\section{Identified shortcomings of Tasking framework}
During the course of the Master thesis, the following shortcomings of the current version of Tasking framework, which is chosen as a computational model for this Master thesis are identified:
\begin{itemize}
\item Tasks from Tasking framework are used in various threads of control as explained in \cref{chap: Progamming model}. At the heart of the Tasking framework is a scheduler which schedules tasks based on priorities and these tasks are non-preemptible at the moment \cite{TaskFr}. This is one of the critical shortcomings in the current version of the Tasking framework as it makes the generated software code which is based on Tasking framework not suitable for hard real-time systems \cite{TempIsolation}. Time-monitoring architectures such as Server-based architecture or Priority-Band architectures listed in \cite{TempIsolation} are the ways to go ahead in case making the code making use of the Tasking framework truly real-time capable. These architectures help in providing isolation of applications i.e., tasks (at least) along three orthogonal dimensional axes: time, space and communication
\item In the current version of the Tasking framework there is no possibility to measure the run-time of the tasks and monitor deadline violations which are mostly caused by WCET overruns of either the task at hand or a higher priority task. This limits the extent of property preservation in the model-to-code transformation step \cite{TempIsolation}. It is of very high importance that the system properties asserted during the analysis and the assumptions made for the analysis to hold are preserved across implementation and execution \cite{EvoRAVCodeAr}\cite{TempIsolation}
\item It is also not possible to measure the execution time of a group tasks which are associated to the single time budget so that the Group Budget is accounted for their collective execution time. This incapability makes the adoption of Server-based architecture in the Tasking framework more difficult
\item In line with the inability to measure the run-time of the tasks from the Tasking framework, Tasking Framework also does not provide any constructs for at least coarse-grained fault detection and fault handling in case of deadline misses
\item 
\end{itemize}

\section{Future Work}
As an enhancement to the current work, it is possible to extend this Master thesis  
\label{section: Future work}


%
%
%\renewcommand{\appendixtocname}{Anhang}
%\renewcommand{\appendixname}{Anhang}
%\renewcommand{\appendixpagename}{Anhang}
\appendix
$  $% !TeX spellcheck = de_DE
%Die Angabe des schlauen Spruchs auf diesem Wege funtioniert nur,
%wenn keine Änderung des Kapitels mittels den in preambel/chapterheads.tex
%vorgeschlagenen Möglichkeiten durchgeführt wurde.
\setchapterpreamble[u]{%
	\dictum[Albert Einstein]{Probleme kann man niemals mit derselben Denkweise lösen, durch die sie entstanden sind.}
}
\chapter{A file structure for the generated code}
\label{chap: File structure}

Considering our running example for code generation in this Master thesis, the generated code is organized into the following files as explained below:

All the data types, event types, interfaces, exception types that are used in the example are stored along with the parameter channel and parameter queue as shown below:

\newpage 

\dirtree{%
	.1 src-gen.
	.2 DatatypesInterfacesEventsAndExceptions.
	.3 include.
	.4 Datatypes.h.
	.4 Exceptions.h.
	.4 FailureEvent.h.
	.4 InterfaceA.h.
	.4 InterfaceB.h.
	.4 ParameterChannel.h.
	.4 ParameterQueue.h.
}

All the constituents of the \texttt{Component\_Callee} are arranged as shown below:

\dirtree{%
	.1 src-gen.
	.2 Component\_Callee.
	.3 AutogeneratedCode.
	.4 include.
	.5 ComponentType\_Callee.h.
	.5 EventEmitterPorts\_Callee.h.h.
	.4 src.
	.5 ComponentType\_Callee.cpp.
	.5 EventEmitterPorts\_Callee.cpp.
	.3 UserCode.
	.4 include.
	.5 ComponentImplementation\_Callee.h.
	.4 src.
	.5 ComponentImplementation\_Callee.cpp.
	.2 Component\_Callee\_impl\_inst\_Instance.
	.3 AutogeneratedCode.
	.4 include.
	.5 ProvidedInterfacePorts\_Callee\_impl\_inst.h.
	.5 ComponentContainer\_Callee\_impl\_inst.h.
	.4 src.
	.5 ProvidedInterfacePorts\_Callee\_impl\_inst.cpp.
	.5 ComponentContainer\_Callee\_impl\_inst.cpp.
}

\newpage

All the constituents of the \texttt{Component\_Caller} are arranged as shown below:
\dirtree{%
	.1 src-gen.
	.2 Component\_Caller.
	.3 AutogeneratedCode.
	.4 include.
	.5 ComponentType\_Caller.h.
	.5 EventReceiverPorts\_Caller.h.
	.5 RequiredInterfacePorts\_Caller.h.
	.4 src.
	.5 ComponentType\_Caller.cpp.
	.5 EventReceiverPorts\_Caller.cpp.
	.5 RequiredInterfacePorts\_Caller.cpp.
	.3 UserCode.
	.4 include.
	.5 ComponentImplementation\_Caller.h.
	.4 src.
	.5 ComponentImplementation\_Caller.cpp.
	.2 Component\_Caller\_impl\_inst\_Instance.
	.3 AutogeneratedCode.
	.4 include.
	.5 ProvidedInterfacePorts\_Caller\_impl\_inst.h.
	.5 ComponentContainer\_Caller\_impl\_inst.h.
	.4 src.
	.5 ProvidedInterfacePorts\_Caller\_impl\_inst.cpp.
	.5 ComponentContainer\_Caller\_impl\_inst.cpp.
}
$  $% !TeX spellcheck = de_DE
%Die Angabe des schlauen Spruchs auf diesem Wege funtioniert nur,
%wenn keine Änderung des Kapitels mittels den in preambel/chapterheads.tex
%vorgeschlagenen Möglichkeiten durchgeführt wurde.
\setchapterpreamble[u]{%
	\dictum[Albert Einstein]{Probleme kann man niemals mit derselben Denkweise lösen, durch die sie entstanden sind.}
}
\chapter{Additional examples}
\label{chap: Extra examples}

%\printindex

\printbibliography

\ifdeutsch
Alle URLs wurden zuletzt am 17.\,03.\,2008 geprüft.
\else
All links were last followed on March 28, 2018.
\fi

\pagestyle{empty}
\renewcommand*{\chapterpagestyle}{empty}
\Versicherung
\end{document}
